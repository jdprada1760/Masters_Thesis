\chapter{Introduction}

A complete physical picture of Dark Matter (DM) is still missing.
This is one of the biggest puzzles to fully understand the composition of our Universe.
So far, its presence can only be measured through its gravitational effect on the surrounding visible matter. 
One of best the astronomical systems that can be used to probe DM on astronomical scales is our own galaxy: the Milky Way (MW).
Probing the DM density field around our galaxy (it's so-called DM halo) can shed light on the nature of DM \cite{Nipoti,ReadMoore} and our
galaxy's formation history \cite{Read1,Read2,Vera-Ciro2011}.\\

One of the most basic features that can be measured in the MW DM halo is its shape. 
Different observational methods have been developed to constrain it. 
They range from the use of Jean's equations applied to stellar kinematics \cite{Loebman2012} to modelling the dynamics of satellite
systems such as the Sagittarius stream and the Large Magellanic Cloud \cite{Vera-Ciro2013,Deg2012,LawMajewski2010}. 
However, different assumptions are made in these studies producing widely different results.
Thus, constraining the density field of the DM halo of the Milky Way remains an open research topic in present-day astronomy.\\ 

Today, computational astrophysics can support all these observationally projects by helping to prove (or disprove) the range of validity of different assumptions \cite{prove,bardeen,Vera-Ciro2011}.
Simulations can also serve to find priors on the expected MW DM halo shape.
However, using simulations comes at a cost.
First, different degrees of realism in the implemented physical models can yield different results.
Second, artifacts can appear due to the always limited numerical resolution. 
For these reasons, the study of simulations of astronomical or cosmological systems, as well as the research for reducing the aforementioned biases of computation, is an important field of study in modern astrophysics.\\

Recently, the growth of computational power and the improvement of numerical models have made possible to perform realistic simulations.
These simulations can trace the non-linear interactions of DM and baryonic (i.e. usual gas and stars) components. 
For instance, the recent development of an state-of-the-art simulation AREPO \cite{arepo} have made possible simulations that were considered impossible a decade ago.
This code has been used to perform the \emph{Auriga Project} \cite{auriga}, which simulates 30 galaxies that reproduce the main Milky Way features such as their stellar masses, rotation curves, star formation rates and metallicities.\\

For this thesis we will use the results from Auriga project \cite{auriga} to study the halo density field of the 30 simulated galaxies.
Specifically, we will measure the shape of the DM halo as a function of its radius and its time evolution.
We will follow the methods presented in a study of the simulation project that preceeded the Auriga Project over 5 years ago \cite{Vera-Ciro2011} that simulated 5 times less galaxies, without any hydrodynamics and at a lower numerical resolution. 
This is the first time that studies of the DM density field are performed with this level of realism.
The simulations in the \emph{Auriga Project} were performed with different hydrodynamical characteristics 
which will also allow us to measure the impact of such differences on the DM halo shape.\\

The results from our study will help to constrain the expected DM density distribution around our galaxy, 
providing a benchmark for all researchers interested in a better understanding of our Galaxy and its 
dark matter distribution.

\section{Constraining the Milky Way's DM Halo}
Talk about Loebman et al., Law-Majewski2010 as examples of different approaches and mention other work's results.

This section must end emphasizing about observational biases.


\section{State of the art on MW simulations}
Introduction of this section is logical continuation of previous section.

Briefly talk about the work of Vera-Ciro2011 and how it is correlated with Vera-Ciro2013.

This section must end emphasizing that we will perform our study based on this previous work on Auriga simulations.

\section{Outline}

Set the outline of the thesis.