\chapter{Introduction}

\section{About Dark Matter}

In observational astronomy it possible to find measurements which are not reconcilable with the current understanding of physical theories. Sometimes, these measurements are the result of instrumental error or incorrect theoretical assumptions. However, since the early beginnings of the 20th century, we have found inconsistencies of observations with the accepted paradigm of physics that challenge the veracity of the Newtionan gravity or even its improved Einsteinian version at astronomical and cosmological scales.\\

Specifically, we observe that objects are moving in a certain way that cannot be explained exclusively by the effect of gravitational forces of visible matter. Observations show that there is an immense lack of matter to explain the gravitational pull reflected on the observed dynamics. To solve this problem there are two important hypothesis to study.\\

One of these hypothesis tries to reconcile these observations with Newtonian/Einsteinian gravity assuming it is completely valid in the astronomical context. Consequently, there must exist some kind of matter that exerts this missing gravitational pull, which we cannot see for some reason. Then, we talk about Dark Matter (DM). Since early as the 1930's, the DM hypothesis has been considered. It is known that pioneer Fritz Zwicky proposed its presence within the Coma cluster of galaxies to explain the cluster attachment despite the huge peculiar velocities of its galaxies that would have made it disperse \cite{Zwicky_1937}. At first, this proposal was not given much attention as there were many other sources of errors to blame and these observations were not widely replicated. Nevertheless, as years passed, more and more inconsistencies could be explained with the use of this DM hypothesis. For instance, it is the case of the famous rotation curves \cite{Faber_and_Gallagher_1979,Rubin_et_al._1980,Persic_et_al._1996} where the amount of visible matter on galaxies could not account for the centripetal acceleration associated with the tangential velocities of stars within it. Later on, with a better understanding of the space-curving effect of matter predicted by general relativity it was possible to measure weak lensing (gravitational distortion of observed objects) through massive clusters of galaxies \cite{Kaiser_and_Squires,Wittman_et_al._2000,Clowe_et_al._2006}. Again, there was not enough visible matter to explain these distortions.\\

Less direct evidence of DM is related to cosmological structure formation and the observed Cosmic Microwave Background Radiation (CMBR) \cite{Blumenthal_et_al._1984}. We know that early stages of the universe were dominated by radiation, which heavily affected visible matter by preventing it to collapse to grow clumps (form structures) from small density perturbations. Specifically, due to radiation, the speed of sound was very close to that of light, meaning that density perturbations must have been huge to not be dispersed as sound waves in this primordial medium. These early-universe density perturbations would be correlated to temperature fluctuations of matter that produced radiation evidenced today around the microwave spectrum due to the redshift from the expansion of the universe. Recently, through precise samplings of the CMBR like the Wilkinson Microwave Anisotropy Probe (WMAP) \cite{WMAP_2003,WMAP_2013} and the Planck Collaboration et al. \cite{Planck_Collaboration_2014,Planck_Collaboration_2016}, it was corroborated that the observed anisotropies are not big enough to explain the primordial gravitational collapse formed the cosmic-scale structures that we observe today. In other words, in the absence of any other kind of pulling force or kind of matter with smaller speed of sound (unaffected by radiation), we would not have the same distribution of collapsed structures (stars, galaxies, clusters) we observe nowadays. \\

Subsequently, the second important hypothesis to reconcile observations and theory studies the possibility that our comprehension of gravity or dynamics is erroneous at astronomical scales. Then we talk about theories of modified gravity or modified dynamics. One of the most successful alternatives of DM is the Modified-Newtonian Dynamical (MOND) theory, which modify Newton's inertia law \cite{Milgrom_1983}. In this case, at some acceleration scale, the gravitatory force is not proportional to the acceleration of an object but to its squared acceleration. If we apply this hypothesis to the previous mismatches, under a precise tuning of parameters, we could obtain the wanted consistency \cite{Sanders_and_McGaugh_2002}.\\

Although these theories may succeed in many galactic-regime predictions \cite{Begeman_et_al._1991,McGaugh_and_de_Blok_1998,McGaugh_2012}, they are extremely well constrained for small-scale regimes by conventional gravity and dynamics. This restricts the freedom with which this models may be adjusted to observations and makes them highly refutable. Some of the biggest problems these theories face are DM remnants that are necessary to completely reconcile predictions and observations \cite{Lokas_2002,McGaugh_2015}. Furthermore, for these theories, the excess of gravitational pull must be centered at the visible-matter distribution, which may not be the case for collisions of galaxies where it is observed that collisionless DM may decouple from visible matter \cite{Clowe_et_al._2006}.\\

For these reasons, DM is the most widely accepted hypothesis to account for observational discrepancies. Nonetheless, a complete physical picture of DM is still missing and it is one of the biggest puzzles to fully understand the composition of our Universe.\\

In the search for the candidate DM, there have been various proposals from Particle Physics, which motivate the existence different kinds of weak-interacting particles through interesting symmetry theories \cite{Kim_1987,Bertone_et_al._2005}.\\

One of the most obvious candidates was the neutrino, a lepton of extremely small mass at rest. Although neutrinos fulfill the basic properties of DM (weak electromagnetic interaction), there were many problems when reconciling with observations. These problems came from the analysis of CMBR where it follows that at early stages of the universe neutrinos would have been relativistic \cite{Bond_et_al._1980,Bertone_et_al._2005}, which would result in a non-hierarchical model of formation. Other problems include constraints on the density of neutrinos \cite{WMAP_2003}, which is not enough to account for the missing observed matter.\\

This kind of very-light DM candidates whose thermal energy significantly affects the proper growth of density anisotropies in the early universe, is known as Hot Dark Matter (HDM). If we assume the existence of a heavier particle as a DM candidate, then these particles would not be relativistic at early stages and therefore they would support a hierarchical model of formation \cite{Blumenthal_et_al._1984,Liddle_and_Lith_1993}. In this case, we refer to this particles as Cold Dark Matter (CDM). Given its consistency with the observable universe, CDM is the most accepted candidate as the principal constituent of DM.\\

\section{Theoretical background for CDM}

Given that CDM has negligible thermal energy, it is usually modeled as a set of gravitating collisionless particles, or as a collisionless self-gravitating Boltzmann fluid described by equation \eqref{eq:Boltzmann}, in the continuous case. As a consequence, neither the fluid/particle nor its interaction force possess a well-defined scale parameter. In other words, by a simple rescaling of position and time, we would arrive to an equivalent system. This characteristic of self-similarity under scale transformations is widely used on theoretical frameworks to simplify the calculations and imply that many properties of DM structures are self-similar. For instance, on one hand Schechter used this argument to analyze the statistics of mass functions in a self-similar universe of DM \cite{Schechter_1976} which has a very precise matching with observations. On the other hand, Bardeen used this self-similarity to analyze the evolution of random Gaussian fluctuations of the DM density field \cite{Bardeen_1986}, resulting in a theoretical framework that is a strong foundation for any work on the analysis of DM structures.\\


\begin{equation}
\frac{d\rho}{dt} = \frac{\partial \rho}{\partial t} +\vec{v}\cdot\frac{\partial\rho}{\partial \vec{x}}
-\frac{\partial \Phi}{\partial \vec{x}}\cdot\frac{\partial\rho}{\partial \vec{v}}.
\label{eq:Boltzmann}
\end{equation}

   
By means of example, using a computational approach, Navarro et al. discovered that DM halos (galactic sized clusters of DM) are ruled by a universal double-power law \eqref{eq:doublePower}, which was independent of the size of the halo or the used cosmology \cite{Navarro_et_al._1996,Navarro_et_al._1997,Power_et_al._2003,Navarro_et_al._2010}. This is expected from the self-similarity properties of DM, although it differs from the previously theoretically predicted single-power law  \cite{Gunn_and_Gott_1972,White_and_Zaritsky_1992}.\\

\begin{equation}
\frac{\rho(r)}{\rho_{crit}} = \frac{\delta_c}{(r/r_s)(1+r/r_s)^2}.
\label{eq:doublePower}
\end{equation}

Interestingly, collisionless models for CDM are a good approximation to model star dynamics or other cosmological objects with negligible cross-section such as Black Holes (BH). Let us take, for example, the Boltzmann transport equation \eqref{eq:Boltzmann} in a convenient set of coordinates to analyze the dynamics of stars in axisymmetric galaxies. If we get rid of the phase-space density function $\rho(\vec{x},\vec{v})$ by averaging in time, we can obtain an expression for stellar dynamics in terms of observable quantities such as the mean velocities $\overline{v_r},\overline{v_z},\overline{v_\phi}$, the velocity correlation matrix $\sigma$, the spatial-density field $\rho$ and the gravitational accelerations $a_r,a_z,a_\phi$. These equations are called Jeans equations in honor of James Jean who was the first to apply this knowledge in the cosmological context \cite{Jeans_1915}. For the case of an axisymmetric stable system \cite{Loebman_et_al._2012}, these equations read:  

\begin{align}
\begin{aligned}
a_r & = \sigma^2_{rr} \frac{\partial \ln \nu}{\partial r} + \frac{\partial \sigma^2_{rr} }{\partial r} + \sigma^2_{rz} \frac{\partial \ln \nu}{\partial z} + \frac{\partial \sigma^2_{rz} }{\partial z} +  \frac{\sigma^2_{rr}}{r}-\frac{\sigma^2_{\phi\phi}}{r}-\frac{\overline{v_\phi}^2}{r}\\
a_z & =  \sigma^2_{rz} \frac{\partial \ln \nu}{\partial r} + \frac{\partial \sigma^2_{rz} }{\partial r} + \sigma^2_{zz}\frac{\partial \ln \nu}{\partial z} + \frac{\partial \sigma^2_{zz} }{\partial z} +  \frac{\sigma^2_{rz}}{r},
\end{aligned}
\label{eq:Jeans}
\end{align}

where the azimutal acceleration $a_{\phi}$ is null for stability reasons and the the spatial density $\rho$ is encoded in the stellar number density distribution $\nu$.\\

In the cosmological context, the gravitational potential $\Phi$ produces the corresponding accelerations. These accelerations (i.e. the left side of Jean's equations) may be obtained by the integration of visible matter, but, as expected, they are underestimated compared to its dynamical effect on $v_{\phi}$ and the matrix $\sigma$ (i.e. the right side of Jean's equations). In this case we have a precise theoretical framework to deduce the missing distribution of DM from the inequality of differently calculated accelerations.\\


\section{Constraining the Milky Way's DM Halo}
So far, DM presence can only be measured through its gravitational effect on the surrounding visible matter. 
One of best the astronomical systems that can be used to probe DM on astronomical scales is our own galaxy: the Milky Way (MW).
Probing the DM density field around our galaxy (it's so-called DM halo) can shed light on the nature of DM \cite{Read_and_Moore_2005,Nipoti_et_al._2007} and our
galaxy history of formation \cite{Read_et_al._2008,Read_et_al._2009,Vera-Ciro_et_al._2011}.\\

DM haloes have two important features that could be observationally constrained. 
On one hand there is the universal radial density profile on equation \eqref{eq:doublePower} \cite{Navarro_et_al._2010}. 
On the other hand, there is the halo shape which encodes the angular dependence of the density field and therefore is related to the halo spin and velocity dispersion. 
According to the hierarchical model of formation, DM halos accrete matter from some preferred directions determined by the local cosmological structures. This is reflected in the shape of the halo which, due to the non-collisional nature of CDM, may be well-characterized by an ellipsoid with three different axes (\textbf{\textit{triaxial}}). Therefore, the halo shape is an important characteristic to \textbf{diagnose} its formation history and its relation with the environment \cite{Bardeen_1986,Vera-Ciro_and_Helmi_2013}.\\

In this sense, it is of special interest to constrain the shape of the only cosmological object of which we have a three-dimensional view from inside: the Milky Way's (MW) DM halo shape. Although, this is a very difficult labour given the observational restrictions, many approaches have been made to iluminate our understanding in this topic. \\

One way to constrain the MW's DM halo shape is to make use of theoretical models that relate the content of matter of our galaxy with the gravitational potential. For example, Loebman et al. \cite{Loebman_et_al._2012} used the axisymmetric Jean's equations \eqref{eq:Jeans} and estimated that, to account for theoretical-observational discrepancies, around 20Kpc the DM halo must be perfectly axisymmetric (spheroid) with an axis ratio of $q_{DM}=0.47 \pm 0.14$ calculated at its isodensity contour. Nevertheless, this axial symmetry is inherited from the use of axisymmetric Jean's equations. 
Although this is a strong assumption, a more general method is much more difficult to implement given the difficulty to obtain the needed data from observations. 
Even authors acknowledge that:

\blockquote{... while it is premature to declare $q_{DM}=0.47 \pm 0.14$ as a robust measurement of the dark matter halo shape, it is encouraging that the simulation is at least qualitatively consistent with SDSS data in so many aspects.}

In this case, rather than considering the precision of the result, we must consider the important advance from considering a spherically-symmetric halo to a more genar case of an spheroid.\\

A stronger approach to constrain the MW's halo shape is to use tracers of the gravitational potential such as the streams generated by close dwarf galaxies that have been deformed by tidal forces from the MW. 
This effect is very important because these forces generated by our halo are very sensible to its shape parameters and would be evidenced in the parameters of these streams \cite{Law_and_Majewski_2009,Law_and_Majewski_2010,Deg_and_Lawrence_2013}. 
In fact, it is known that a constantly axisymmetric halo cannot simultaneously explain all the features of the Sagittarius dwarf galaxy deformation \cite{Law_and_Majewski_2009}. \\

In this context Law and Majewski 2010 proposed an analytical model of the MW containing a triaxial halo characterized by the axial ratios and orientation of its isopotential contour. 
Given all these parameters, the Sagittarius stream was simulated and evolved forward and backwards in time for various choices of the halo parameters. 
The shape parameters that best fitted a detailed study of the observational properties of the Sagittarius stream, was found at a minor-to-major axis ratio $(c/a)_{\Phi}=0.72$ and intermediate-to-major axis ratio $(b/a)_{\Phi}=0.99$. The minor axis of this triaxial halo was found to be pointing in some direction contained in the galactic disk plane. \\

This model took the leap from an axisymmetric DM halo to a the more general model of a triaxial halo.
Nevertheless, the coexistence of a triaxial DM halo and an axisymmetric galactic disk is debated by Cold Dark Matter (CDM) models \cite{Debattista_et_al._2008}. 
Specifically, it is expected that the DM and gas distributions are correlated in the sense that matter is accreted from the same cosmic structures as is DM. 
Therefore, visible matter and DM must have aligned angular momenta to certain extent, i.e. the minor axes of the galactic disk and the DM halo must be aligned. 
Furthermore, to guarantee the stability of the galactic disk, DM distributions should be partially axisymmetric in the presence of the axisymmetric disk potential \cite{Ostriker_and_Peebles_1973}.\\

Law and Majewski comment in their paper: 

\blockquote{... by no means do they (results) represent best-fit models in a statistical sense. Therefore, the predictions made cannot be considered exclusive or definitive but serve to guide where future observations could focus to distinguish between various models.}

Studies of this kind are by nature non-exact due to the difficulty in obtaining precise information from observations.
In observations, we take 2-dimensional snapshots of the sky and loose resolution of the radial density field due to screening of matter. 
This makes the process of obtaining  a three-dimensional view of a cosmological-scale object an extremely difficult endeavour. 
Furthermore, we can determine radial velocities with Doppler effect, but there is no obvious way of obtaining tangential velocities. 
Bearing this in mind, any study which is sensible to detailed observational parameters for obtaining non-direct measurements of the DM density field, will be either non-conclusive for reasonable-difficulty models, or must be extremely sophisticated to achieve a significantly exact result.\\

\section{State of the art of MW simulations}
To address the observational difficulties, there is an important field in astrophysics consisting in the modeling of the non-linear behaviour of matter at cosmological scales. 
This is with the objective of numerically simulating the universe at a wide range of scales and produce systems of which we have full control of their parameters. 
In this sense, a computer may become a virtual cosmological laboratory, where we may run different experiments having control over their initial conditions, and in this way support or verify theoretical frameworks. \\


Modern high-resolution cosmological simulations are usually restricted to modeling DM as a non-collisional fluids as in equations \eqref{eq:Boltzmann} and gas as an Eulerian collisional fluid as in equations \eqref{eq:Euler}.  Efficiently solving these systems of non-linear equations is an intricate puzzle and is still an open and improving field of research. 
Difficulties in the modeling of these fluids arise from numerical instabilities and the wide range of values that quantities may take in the cosmological context, which may quickly expand in several orders of magnitude, becoming problematic numerical discontinuities. Depending on the current computing power, these simulations are limited to some resolution, which is adjustable to the specific objective of the research.\\


\begin{align}
\begin{aligned}
&\frac{d\rho}{dt} + \rho \vec{\nabla}\cdot\vec{v} = 0\\
&\frac{d\vec{v}}{dt} = -\frac{\vec{\nabla}P}{\rho} - \vec{\nabla} \Phi \\
&\frac{du}{dt} = -\frac{P}{\rho}\vec{\nabla}\cdot\vec{v} - \frac{{\Lambda(u,\rho)}}{\rho}\\
& P = (\gamma -1 )\rho u
\end{aligned}
\label{eq:Euler}
\end{align}


In a historical context, numerical astrophysics have experienced a parallel growth with computing power and the improvement of numerical methods. 
As early as 1960's, Miller \& Pendergast \cite{Miller_and_Prendergast_1968} and Hohl \& Hockney  \cite{Hohl_and_Hockney_1969} used small bidimensional N-body simulations \cite{libro DM} to try to recover the stable spiral shape of galaxies like MW. At this point, the computational power was not enough to even attempt a solution of collisional fluids like gas. Therefore, to simulate disipation effects of gas clouds, they emulated the temperature as a randomized field of peculiar velocites applied over their punctual masses and implemented a cooling process in which some random fraction of particles lose energy \cite{Miller_et_al._1970}. With this work, and some insight from Jerry Ostriker and James Peebles \cite{Ostriker_and_Peebles_1973}, it was demonstrated that galactic disks cannot be stable on their own and need some sort of additional radial pull (DM) \cite{libro DM}. In these simulations it was also verified that spiral branches of galaxies are a consequence of the propagation of small density fluctuations driven by dissipation of energy by gas clouds. Parallel to this work, it was also demonstrated by Toomre and Toomre \cite{Toomre_and_Toomre_1972} that these disruptions could also be caused by tidal forces of close encounters between galaxies.\\

Even with the advances of computational power following Moore's law, the numerical methods used to solve simulations quickly became obsolete year by year and needed to be pushed forward to an optimization of the computational resources to serve the purposes of astrophysicists. This co-evolution between numerical methods and astrophysics together with the exponential growth of computational power, eventually made possible the performance of Cosmological-sized DM-only simulations such as Millennium which could reproduce the observed cosmological structures at a wide range of extra-galactic scales \cite{Millennium}. \\ 

In this context, the analysis of baryonic matter was usually imprinted through the use of semi-analytic methods. That is, they took the numerical output of a DM-only simulation and performed analytical calculations to follow the evolution of the properties of point-mass galaxies and match the statistics of observable galaxies. \cite{Lemson_et_al._2006,De_Lucia_et_al._2006}. Although these semi-analytical methods did not trace the evolution of baryons to a level of resolution that allowed the study of galactic dynamics, it was the most realistic model that could be performed in cosmological-sized simulations and are still a good tool to understand the specifics of galaxy formation and even constrain the parameters of some cosmological models. By this time, numerical methods trying to actually simulate baryons to galactic-scale resolutions, suffered from a process of over-cooling in which gas collapsed too quickly and did not produce stable galaxy-sized disk like we observe today \cite{Katz_and_Gunn_1991}.\\

This over-cooling process that affected galactic-sized simulations was understood to be a consequence of not having into account events of energy redistribution that stopped the rapid collapse. This energy redistribution produced thermic and kinematic pressure that fought the gravitational pressure. The principal sources of this energy transport are now known to be supernovae (SN) explosions, radiation from cosmic rays and the Active Galactic Nucleus (AGN), which are usually referred to as stellar feedback, radiation pressure and AGN feedback, respectively. They enclose modern physics processes which cannot be fully simulated and must be estimated with general recipes with some free parameters. For instance, stars are seen as a point mass and when they reach some density or arrive to certain age, they may result in a SN explosion.  This explosion process of energetic feedback is simulated by isotropically liberating some amount (free parameter) of kinetic energy (radial velocity recoil) as well as some amount of thermal energy (thermal energy), to the surrounding gas cells. \\

 A decade ago, these feedback processes were not as well understood as they are today. For this reason, the simulation of MW-like galaxies in the whole cosmological context is often subject to simplifications that produce unrealistic galaxies studied as boundary cases \cite{Bryan(unrealistic galaxies)}. As a matter of fact, it has been possible only until recently the simulation of realistic galactic-sized objects like the MW, tracing the evolution of normal matter alongside with DM with an exceptional accuracy that allows the study of galactic dynamics \cite{Auriga}.\\
 
In this sense, stat-of-the-art simulations, and specifically MW-like simulations, represent an important tool to comprehend specific features of out universe in a wide range of scales and, as Law \& Majewski said, "serve to guide where observations could focus to distinguish between various theoretical models".
 
 


\section{Synergy between Theory, Observations and Simulations }
%In the CDM paradigm, we have fully theoretical studies \cite{Bardeen_1986,Schechter_1976} principally focused in the analysis of Gaussian random fields and the properties of self-similarity that DM must possess. 
%These theoretical frames are then supported by CDM simulations and, if possible, by observations. 
%In fact, these theories are usually thoroughly verified and complemented through simulations, given their convenient malleability, before being directly applied to observations.\\% Redactar mejor (simbiosis entre teor[ia analitica, simulaciones numericas, y observaciones.

The synergy between cosmological simulations, theory and observations is well exemplified in the work of Vera-Ciro et al. (2011-2013). In 2011, Vera-Ciro et al. studied the shape of a set of four MW-like DM-only galaxies from Aquarius simulations \cite{aquarius}, with the objective of complementing the predictions of the CDM paradigm. 
For instance, previous theoretical studies of halo shapes were usually restricted to the individual halos and their characteristics at redshift 0, neglecting information about their history of formation.
Intuitively, it is expected that halo shapes vary with the radius taking into account that accretion occurs at progressively bigger radii in history and that the gravitational potential that affects infalling matter may deppend on the radius.
Aditionally, the hierarchical model of structure formation predicted that the halo shape must be influenced by the environmental cosmic structures that determined the preferred directions of matter infall \cite{Tormen_et_al._1997,Colberg_et_al._1999}.\\

In this way, Vera-Ciro et al. showed in 2011 that, due to the increase in the cross section of the halos which contributes to the scattering of particles and randomization of orbits, at later stages and bigger radii, DM halos become more spherical.
Furthermore, the radial evolution of the halo axial ratios that determine their shape is in fact correlated with its historic shape and its environment. 
These results helped to obtain more insight about the galactic dynamics of formation and also suggested some guidelines to improve Law and Majewski 2010 study, taking into account that DM halos shapes are not as constant as assumed in observational models.\\

In 2013, Vera-Ciro and Helmi proposed an improved study based on the one performed by Law and Majewski in 2010. Taking into account that the DM halo shape may not be constant, they modeled a DM halo as perfectly oblate at inner regions, which transitioned smoothly to a triaxial halo in the outer-skirts. With this, minor-axis alignment inconsistency of previous static-shape models is solved. They found that this halo, sampled at its isopotential contours, is triaxial in the outer skirts with a medium-to-major axis ratio of $0.9$ and minor-to-major axis of $0.8$, which is still very oblate regarding the more prolate CDM predictions.\\

Besides solving some inconsistencies from previous studies with the expected predictions, Vera-Ciro et al. demonstrated that small changes in the halo triaxiality, such as its variation with radius, are important. This study comes from DM-only simulations and succeeded at improving previous constraints on the shape of our MW halo from a better understanding of DM. Taking into account the new and improved hydrodynamical simulations, which can model baryons alongside DM, our principal motivation in this work is to use these runs to validate previous knowledge about DM, as well as obtaining new insights on the DM-baryons relation. Specifically, we want to verify how the shape of a DM halo is affected by radius, redshift and presence of visible matter.\\

%
%Briefly talk about the work of Vera-Ciro2011 and how it is correlated with Vera-Ciro2013.

\section{Outline}
This thesis is organized as follows: in Chapter 2 we present details about the simulations in which we perform our calculations, as well as the specifics of our shape-calculating method. In Chapter 3 we present our results. And Finally, in Chapter 4 we present our conclusions.  
