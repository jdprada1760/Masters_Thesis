\chapter{Introduction}

\section{About Dark Matter}

In the field of observational astronomy it is very plausible to encounter ourselves with indirect or direct measurements which are not reconcilable with the current understanding of physical theories. Very often, these intriguing measurements are the result of inaccuracy in the measurement device or the asumption of some erroneous or non-precise premises. However, since the early beginings of the 20th century, we have found inconsistencies of observations with the accepted paradigm of physics, which have been reliably proven not to arise from any errors in the premises or devices. These inconsistent measurements principally question the veracity of the Newtionan gravity or even its improved Einstenian version at astronomical and cosmological scales.\\

The principal problem of these inconsistencies can be resumed to the fact that we observe that objects are moving in a certain way that cannot be explained exclusively by the effect of gravitational forces of visible matter. In fact, there is very strong historical evidence that this gravitational effect is heavily \textit{underestimated} if we only take into account the effect of light-interacting matter. To solve this problem without blaming the measurement instruments or the used premises, which have been consistently ruled out as the sources of these discrepancies, there are two important hypothesis to study.\\

One of these hypothesis, which is the most simple to understand, tries to reconcile these observations with Newtonian gravity, or even more, General Relativity, assuming it is completely valid in the cosmological context. Consequently, if there is nothing wrong with the theoretical framework, the premises, nor the instruments, and gravitational effects cannot be explained solely with the matter that we observe, then there must exist some kind of matter that exerts this missing gravitational pull, which we cannot see for some reason. We talk about Dark Matter (DM).\\

Since early as the 1930's, the DM hypothesis has been considered. It is known that Fritz Zwicky proposed the presence of some kind of DM within the Coma cluster of galaxies to explain the cluster attachment despite the galaxie's huge velocities that would have made them scape \cite{Fritz}. At first, this proposal was not given much attention as there were many other sources of errors to blame, and these observations were not widely replicated. Nevertheless, as years passed, more and more inconsistencies could be explained with the use of this DM hypothesis. It is the case of the famous rotation curves \cite{rotationCurves} where the ammount of visible matter on galaxies could not account for the centripetal acceleration associated with the tangential velocities of stars within a it. It is, in essence, the argument of Zwicky. Later on, with a better understanding of the space-curving effect of matter predicted by general relativity and the improvement of precision in telescopes, it was posible to measure weak lensing (gravitational distortion of observed objects) through massive clusters of galaxies \cite{weak_Lensing}. Again, this effect was estimated to be much more dim compared to the actual result of observations, suggesting the existence of DM. This is, in fact, a naturally different approach than those of velocity-curves analysis.\\

Less direct evidence of DM include arguments related to structure formation and the observed Cosmig Microwave Background Radiation (CMBR) \cite{structure formation}. On one hand, the structure formation argument is based in the fact that the early stages of the universe were dominated by radiation, which heavily affects visible matter by preventing it to grow clumps from small density anisotropies. In fact, the speed of sound at the early stages of the universe is very close to that of the speed of light, meaning that density anisotropies must be quite large and localized for them to not be dispersed as sound waves. These anisotropies would be evidenced on the CMBR as temperature fluctuations. It has been thoroughly corroborated that these anisotropies are not present on a very precise sampling of the CMBR like WMAP \cite{WMAP} and PLANCK \cite{PLANCK} In the absence of any other kind of pulling force or sort of matter with smaller speed of sound, such as DM, we would not have the same distribution of collapsed structures we observe today, like stars, galaxies, clusters, etc. as these density fluctuations would have been dispersed. \\


Subsequently, the second important hypothesis to reconcile observations with our theoretical framework is of oposing nature than that of DM. In this case, we study the posibility that there General Theory of Relativity is not accurate and is in fact underestimated at cosmological scales. We talk in this case of Modified-Newtonian Dynamical (MOND) theories. This is a natural consequence of assuming that the universe is what we observe and there is no such thing as non-visible matter. If we apply this hypothesis to the problems solved with the assumption of the existence of DM, under a precise tunning of parameters of these new theories, we could obtain the seeked consistency. However, the problem arises when we try to satisfy this consistency for all observational discrepancies. In the case of DM, many of the observational mismatches with theory may be solved and it is possible to make predictions. In the case of MOND theories, many results as rotation curves may be emulated to certain extent \cite{MOND}, but given its strong claims, it is some times imposible to reconcile with observations \cite{MOND NO}. In fact, these teories are extremely well constrained for small-scale regimes, which restricts the freedom with which this models may be adjusted to observations, to such an extent that some times a remanent of DM is needed to completely concile the theory with observation \cite{DM remanent.}.\\

For these reasons, DM is the most widely accepted hypothesis to account for these observational discrepancies. Nonetheless, a complete physical picture of DM is still missing and it is one of the biggest puzzles to fully understand the composition of our Universe. Many appealing candidates may be found in the context of Particle Physics, which motivate the existence different kinds of weak-interacting particles through interesting symmetry-seeking theories \cite{particle physics.}.\\

\section{Theoretical background for DM}


\section{Constraining the Milky Way's DM Halo}
So far, DM presence can only be measured through its gravitational effect on the surrounding visible matter. 
One of best the astronomical systems that can be used to probe DM on astronomical scales is our own galaxy: the Milky Way (MW).
Probing the DM density field around our galaxy (it's so-called DM halo) can shed light on the nature of DM \cite{Nipoti,ReadMoore} and our
galaxy's formation history \cite{Read1,Read2,Vera-Ciro2011}.\\

DM haloes have two important features that could be constrained. 
On one hand there is the density profile, which has been demonstrated to follow an approximately universal model \cite{Navarro et al. 2010}. 
On the other hand, there is the halo shape, which is directly related to its spin. 
According to the hierarchical model of structure formation, due to the anisotropic history of accretion DM haloes are triaxial and therefore, their shape and spin are important characteristics to \textbf{diagnose} their formation history \cite{Bardeen et al. 1986}.\\

In this sense, it is of special interest to constrain the DM halo shape of the only cosmological object of which we have a tridimensional view from inside: our Milky Way. However, this is a very difficult labour given the observational restrictions of observation. Many approaches have been made to constrain the MW's DM shape. One of them is to make use of theoretical models that relate the content of matter of our galaxy with the gravitational potential.\\

For example, Loebman et al. \cite{Loebman et al 2012} used the axisymmetric Jean's equations \cite{Jean 2015} that relate the stellar content of our galaxy, with the radial and axial accelerations. 
The observed accelerations cannot be completely explained by visible matter only and DM presence is needed. Loebman et al. estimated that, around 20Kpc, the DM halo must be perfectly oblate with axis ratio of $q_{DM}=0.47 \pm 0.14$ to account for this discrepancy.\\

Nevertheless, the axial symmetry that characterizes this halo is inherited from the use of axisymmetric Jean's equations. 
Although this is a strong asumption, a more general theoretical background is much more difficult to implement given the difficulty to obtain the needed data from observations. 
Even authors aknowledge that "... while it is premature to declare $q_{DM}=0.47 \pm 0.14$ as a robust measurement of the dark matter halo shape, it is encouraging that the simulation is at least qualitatively consistent with SDSS data in so many aspects".
This demonstrates that this field of study is still very young and any calculated constraint may lead us to a better understanding of our MW's DM halo shape.\\

A more common and strong approach is to use the streams of close dwarf galaxies that have been deformed by the gravitational potential of the MW. 
This effect is very important because the torca generated by the anisotropy in our halo is sensible to its parameters and thus, these streams are strong evidence to constrain the shape of our MW's DM halo \cite{See Law-Majewski references}. 
In fact, it is known that a static axisymmetric halo cannot simultaneously explain all the features of the Sagittarius leading arm \textbf{citaa}. \\

In this context Law and Majewski 2010 proposed an analytical model of the MW consisting of a fixed analytical gravitational potential formed by a Miyamoto-Nagai \cite{Miyamoto-Nagai 1985} disk, a Hernquist spheroid and a logarithmic halo. 
This halo is triaxial and is characterized by its axial ratios and orientation. 
Given all these parameters, the Sagittarius stream was simulated and evolved forward and backwards in time for various choices of the halo parameters. 
The best fit, compared to a detailed study of the observational properties of the Sagittarius stream, was found at a minor/major axis ratio $(c/a)_{\Phi}=0.72$ and intermediate/major axis ratio $(b/a)_{\Phi}=0.99$. The minor axis of this triaxial halo was found to be pointing in some direction contained in the galactic disk plane. \\

This sophisticated model succeeded at simultaneously reproducing the radial velocity and angular position trends of the Sagittarius leading arm, which were troublesome to model with simpler approaches. 
Nevertheless, the coexistence of a triaxial DM halo and an axisymmetric galactic disk is not supported by Cold Dark Matter (CDM) models \cite{Debattista et al. 2008}. 
Specifically, it is expected that the DM and gas distributions are correlated in the sense that matter is accreted similarly as is DM, being the interaction properties the principal difference in the behaviour. 
Having this into account, gas and DM must have aligned angular momenta to certain extent because all kinds of matter are expected to be accreted from the same cosmic structures. 
In other words, it is expected for minor axes to be rather aligned. 
Furthermore, due to stability reasons and a historical interaction, the matter distribution should be non-axisymmetric in the presence of a non-axisymmetric halo potential.\\

Law \& Majewski comment in their paper: "... by no means do they (results) represent best-fit models in a statistical sense. Therefore, the predictions made cannot be considered exclusive or definitive but serve to guide where future observations could focus to distinguish between various models.". 
Particularly, this discrepancy with the current CDM paradigm may be a feature of the specific model. 
Other important observational constraints were dismissed in this study, such as the non-symmetric influence of the Large Magellanic Cloud (LMC). 
This feature may obviate the triaxial halo and produce a more CDM-consistent model. 
However, observationally obtaining the detailed information of the LMC needed for this kind of research is extremely difficult.\\

Studies of this kind are by nature non-conclusive(deterministic?, not so conclusive?) due to the dificulty in obtaining precise information from observations. 
In observations, we take 2-dimensional snapshots of the sky and therefore, we loose resolution of the radial density field due to screening. 
This makes the process of obtaining  a tridimensional view of a cosmological-scale object an extremely difficult endeavour. 
Furthermore, we can determine radial velocities with dopler effect, but there is no obvious way of obtaining tangencial velocities. 
Bearing this in mind, any study which is sensible to very detailed observational parameters for obtaining non-direct measurements of the DM density field will be either non-conclusive for reasonable-difficulty models (as is the process of constraining), or must be extremely sophisticated to achieve a significantly conclusive result.\\

\section{State of the art on MW simulations}
To address this specific difficulty of obtaining information from observations, there is a vast and important field consisting in the modeling of the non-linear behaviour of matter. 
This is with the objective of numerically simulating the universe at a wide range of scales and produce consistent systems of which we have  full control of their parameters at all stages. 
In this sense, a computer may become a virtual cosmological-scale laboratory, where we may run an experiment having full control over its initial conditions to compare different outputs and support theoretical frameworks. \\

In this sense, in the CDM paradigm, we have fully theoretical studies \cite{Bardeen et al. 1986,Schechter} principally focused in the analysis of Gaussian random fields and the properties of self-similarity that DM must possess. 
These theoretical frames are then supported by CDM simulations \cite{Cualquier estudio de CDM hace referencia a esos pilares} and, if possible, by observations. 
In fact, these theories are usually thoroughly verified and complemented through simulations, given their convenient malleability, before being directly applied to observations.\\ Redactar mejor (simbiosis entre teor[ia analitica, simulaciones numericas, y observaciones.

One good example of this synergy between analytical models, numerical simulations and observational data is evidenced in the work of Vera-Ciro et al. (2011-2013). In 2011, Vera-Ciro et al. studied the shape of a set of four simulated MW-like DM galaxies with the objective of complementing the predictions of the CDM paradigm. Specifically, the hierarchical model of structure formation predicts that the halo shape is correlated with the environment given that it determines the structures from which it accretes matter \cite{referencias sobre relacion entre forma y entorno}. However, theoretical studies are restricted to the correlations at reshift 0 and do not say much about history of formation. Intuitively, it is expected that halo shapes vary with the radius taking into account that accretion occurrs at progresively bigger radii in history and that the cosmic structures that determine environments evolve during that time. Due to the collisionless nature of DM, inner shells can interact with outer shells only in a gravitational way. This means that the historical shape is somewhat conserved in the radial shape profile.\\

Vera-Ciro et al.  showed in 2011 that the radial profile of the halo shape is indeed correlated with its accretion history and environment. Furthermore, due to the increase in the cross section of the halos, which contributes to the scattering of particles(?), at later stages and bigger radii, they become more oblate/spherical. This results helped to obtain more insight about the galactic dynamics of formation and also suggested some guidelines to improve Law \& Majewski 2010 study.\\

In 2013, Vera-Ciro et al. proposed an improved study based on the one by Law \& Majewski in 2011. Vera-Ciro et al. proposed a halo that is perfectly oblate at inner regions and transitions smoothly to a triaxial halo in the outter-skirts. With this, the angular momentum inconsistency of this constraint with the CDM model is solved. They found that this halo is triaxial in the outter skirts with a medium-to-major axis ratio of $0.9$ and minor-to-major axis of $0.8$, which is still very oblate regarding the CDM predictions.\\

However, even when this study solves some inconsistencies with the expected predictions, it demonstrated that small perturbations are important. That is, even when the Sagittarius stream samples the gravitational potential at the outter parts of the halo, where the shape of the inner regions should not be so important, the outter shape is affected to compensate for the change in the regions from inside. This effect takes our attention to a relegated topic: the LMC. In fact, Vera-Ciro et al. demonstrated that the change in shape from the inner regions produces a torque comparable to that of the LMC, which should be taken into account in further researches.\\

%
%Briefly talk about the work of Vera-Ciro2011 and how it is correlated with Vera-Ciro2013.

This section must end emphasizing that we will perform our study based on this previous work on Auriga simulations.

\section{Outline}

Set the outline of the thesis.
