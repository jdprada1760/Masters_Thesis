\chapter{Introduction}

A complete physical picture of Dark Matter (DM) is still missing.
This is one of the biggest puzzles to fully understand the composition of our Universe.
So far, its presence can only be measured through its gravitational effect on the surrounding visible matter. 
One of best the astronomical systems that can be used to probe DM on astronomical scales is our own galaxy: the Milky Way (MW).
Probing the DM density field around our galaxy (it's so-called DM halo) can shed light on the nature of DM \cite{Nipoti,ReadMoore} and our
galaxy's formation history \cite{Read1,Read2,Vera-Ciro2011}.\\

One of the most basic features that can be measured in the MW DM halo is its shape. 
Different observational methods have been developed to constrain it. 
They range from the use of Jean's equations applied to stellar kinematics \cite{Loebman2012} to modeling the dynamics of satellite
systems such as the Sagittarius stream and the Large Magellanic Cloud \cite{Vera-Ciro2013,Deg2012,LawMajewski2010}. 
However, different assumptions are made in these studies producing widely different results.
Thus, constraining the density field of the DM halo of the Milky Way remains an open research topic in present-day astronomy.\\ 

Today, computational astrophysics can support all these observationally projects by helping to prove (or disprove) the range of validity of different assumptions \cite{prove,bardeen,Vera-Ciro2011}.
Simulations can also serve to find priors on the expected MW DM halo shape.
However, using simulations comes at a cost.
First, different degrees of realism in the implemented physical models can yield different results.
Second, artifacts can appear due to the always limited numerical resolution. 
For these reasons, the study of simulations of astronomical or cosmological systems, as well as the research for reducing the aforementioned biases of computation, is an important field of study in modern astrophysics.\\

Recently, the growth of computational power and the improvement of numerical models have made possible to perform realistic simulations.
These simulations can trace the non-linear interactions of DM and baryonic (i.e. usual gas and stars) components. 
For instance, the recent development of an state-of-the-art simulation AREPO \cite{arepo} have made possible simulations that were considered impossible a decade ago.
This code has been used to perform the \emph{Auriga Project} \cite{auriga}, which simulates 30 galaxies that reproduce the main Milky Way features such as their stellar masses, rotation curves, star formation rates and metallicities.\\

For this thesis we will use the results from Auriga project \cite{auriga} to study the halo density field of the 30 simulated galaxies.
Specifically, we will measure the shape of the DM halo as a function of its radius and its time evolution.
We will follow the methods presented in a study of the simulation project that preceeded the Auriga Project over 5 years ago \cite{Vera-Ciro2011} that simulated 5 times less galaxies, without any hydrodynamics and at a lower numerical resolution. 
This is the first time that studies of the DM density field are performed with this level of realism.
The simulations in the \emph{Auriga Project} were performed with different hydrodynamical characteristics 
which will also allow us to measure the impact of such differences on the DM halo shape.\\

The results from our study will help to constrain the expected DM density distribution around our galaxy, 
providing a benchmark for all researchers interested in a better understanding of our Galaxy and its 
dark matter distribution.

\section{Constraining the Milky Way's DM Halo}
According to the hierachichal model of structure formation, DM halos are a commmon and important feature to understand galactic and extra-galactic sized objecs. 
However, performing direct measurements of DM is very difficult given its elusive nature. 
Therefore, constraining the main characteristics of DM halos is an important field of study not only in the context of astrophysics but for any other area interested in obtaining some insight in the fundamental enigma of DM.\\

DM haloes have two important features that can be constrained. 
On one hand there is the density profile, which has been demonstrated to follow an approximately universal model \cite{Navarro et al. 2010}. 
On the other hand, there is the halo shape, which is directly related to its spin. 
According to the hierarchical model of structure formation, due to the anisotropic history of accretion DM haloes are triaxial and therefore, their shape and spin are important characteristics to \textbf{diagnose} their formation history \cite{Bardeen et al. 1986}.\\

In this sense, it is of special interest to constrain the DM halo shape of the only cosmological object of which we have a tridimensional view from inside: our Milky Way. However, this is a very difficult labour given the observational restrictions of observation. Many approaches have been made to constrain the MW's DM shape. One of them is to make use of theoretical models that relate the content of matter of our galaxy with the gravitational potential.\\

For example, Loebman et al. \cite{Loebman et al 2012} used the axisymmetric Jean's equations \cite{Jean 2015} that relate the stellar content of our galaxy, with the radial and axial accelerations. 
The observed accelerations cannot be completely explained by visible matter only and DM presence is needed. Loebman et al. estimated that, around 20Kpc, the DM halo must be perfectly oblate with axis ratio of $q_{DM}=0.47 \pm 0.14$ to account for this discrepancy.\\

Nevertheless, the axial symmetry that characterizes this halo is inherited from the use of axisymmetric Jean's equations. 
Although this is a strong asumption, a more general theoretical background is much more difficult to implement given the difficulty to obtain the needed data from observations. 
Even authors aknowledge that "... while it is premature to declare $q_{DM}=0.47 \pm 0.14$ as a robust measurement of the dark matter halo shape, it is encouraging that the simulation is at least qualitatively consistent with SDSS data in so many aspects".
This demonstrates that this field of study is still very young and any calculated constraint may lead us to a better understanding of our MW's DM halo shape.\\

A more common and strong approach is to use the streams of close dwarf galaxies that have been deformed by the gravitational potential of the MW. 
This effect is very important because the torca generated by the anisotropy in our halo is sensible to its parameters and thus, these streams are strong evidence to constrain the shape of our MW's DM halo \cite{See Law-Majewski references}. 
In fact, it is known that a static axisymmetric halo cannot simultaneously explain all the features of the Sagittarius leading arm \textbf{citaa}. \\

In this context Law and Majewski 2010 proposed an analytical model of the MW consisting of a fixed analytical gravitational potential formed by a Miyamoto-Nagai \cite{Miyamoto-Nagai 1985} disk, a Hernquist spheroid and a logarithmic halo. 
This halo is triaxial and is characterized by its axial ratios and orientation. 
Given all these parameters, the Sagittarius stream was simulated and evolved forward and backwards in time for various choices of the halo parameters. 
The best fit, compared to a detailed study of the observational properties of the Sagittarius stream, was found at a minor/major axis ratio $(c/a)_{\Phi}=0.72$ and intermediate/major axis ratio $(b/a)_{\Phi}=0.99$. The minor axis of this triaxial halo was found to be pointing in some direction contained in the galactic disk plane. \\

This sophisticated model succeeded at simultaneously reproducing the radial velocity and angular position trends of the Sagittarius leading arm, which were troublesome to model with simpler approaches. 
Nevertheless, the coexistence of a triaxial DM halo and an axisymmetric galactic disk is not supported by Cold Dark Matter (CDM) models \cite{Debattista et al. 2008}. 
Specifically, it is expected that the DM and gas distributions are correlated in the sense that matter is accreted similarly as is DM, being the interaction properties the principal difference in the behaviour. 
Having this into account, gas and DM must have aligned angular momenta to certain extent because all kinds of matter are expected to be accreted from the same cosmic structures. 
In other words, it is expected for minor axes to be rather aligned. 
Furthermore, due to stability reasons and a historical interaction, the matter distribution should be non-axisymmetric in the presence of a non-axisymmetric halo potential.\\

Law \& Majewski comment in their paper: "... by no means do they (results) represent best-fit models in a statistical sense. Therefore, the predictions made cannot be considered exclusive or definitive but serve to guide where future observations could focus to distinguish between various models.". 
Particularly, this discrepancy with the current CDM paradigm may be a feature of the specific model. 
Other important observational constraints were dismissed in this study, such as the non-symmetric influence of the Large Magellanic Cloud (LMC). 
This feature may obviate the triaxial halo and produce a more CDM-consistent model. 
However, observationally obtaining the detailed information of the LMC needed for this kind of research is extremely difficult.\\

Studies of this kind are by nature non-conclusive(deterministic?, not so conclusive?) due to the dificulty in obtaining precise information from observations. 
In observations, we take 2-dimensional snapshots of the sky and therefore, we loose resolution of the radial density field due to screening. 
This makes the process of obtaining  a tridimensional view of a cosmological-scale object an extremely difficult endeavour. 
Furthermore, we can determine radial velocities with dopler effect, but there is no obvious way of obtaining tangencial velocities. 
Bearing this in mind, any study which is sensible to very detailed observational parameters for obtaining non-direct measurements of the DM density field will be either non-conclusive for reasonable-difficulty models (as is the process of constraining), or must be extremely sophisticated to achieve a significantly conclusive result.\\

\section{State of the art on MW simulations}
To address this specific difficulty of obtaining information from observations, there is a vast and important field consisting in the modeling of the non-linear behaviour of matter. 
This is with the objective of numerically simulating the universe at a wide range of scales and produce consistent systems of which we have  full control of their parameters at all stages. 
In this sense, a computer may become a virtual cosmological-scale laboratory, where we may run an experiment having full control over its initial conditions to compare different outputs and support theoretical frameworks. \\

In this sense, in the CDM paradigm, we have fully theoretical studies \cite{Bardeen et al. 1986,Schechter} principally focused in the analysis of Gaussian random fields and the properties of self-similarity that DM must possess. 
These theoretical frames are then supported by CDM simulations \cite{Cualquier estudio de CDM hace referencia a esos pilares} and, if possible, by observations. 
In fact, these theories are usually thoroughly verified and complemented through simulations, given their convenient malleability, before being directly applied to observations.\\ Redactar mejor (simbiosis entre teor[ia analitica, simulaciones numericas, y observaciones.

One good example of this synergy between analytical models, numerical simulations and observational data is evidenced in the work of Vera-Ciro et al. (2011-2013). In 2011, Vera-Ciro et al. studied the shape of a set of four simulated MW-like DM galaxies with the objective of complementing the predictions of the CDM paradigm. Specifically, the hierarchical model of structure formation predicts that the halo shape is correlated with the environment given that it determines the structures from which it accretes matter \cite{referencias sobre relacion entre forma y entorno}. However, theoretical studies are restricted to the correlations at reshift 0 and do not say much about history of formation. Intuitively, it is expected that halo shapes vary with the radius taking into account that accretion occurrs at progresively bigger radii in history and that the cosmic structures that determine environments evolve during that time. Due to the collisionless nature of DM, inner shells can interact with outer shells only in a gravitational way. This means that the historical shape is somewhat conserved in the radial shape profile.\\

Vera-Ciro et al.  showed in 2011 that the radial profile of the halo shape is indeed correlated with its accretion history and environment. Furthermore, due to the increase in the cross section of the halos, which contributes to the scattering of particles(?), at later stages and bigger radii, they become more oblate/spherical. This results helped to obtain more insight about the galactic dynamics of formation and also suggested some guidelines to improve Law \& Majewski 2010 study.\\

In 2013, Vera-Ciro et al. proposed an improved study based on the one by Law \& Majewski in 2011. Vera-Ciro et al. proposed a halo that is perfectly oblate at inner regions and transitions smoothly to a triaxial halo in the outter-skirts. With this, the angular momentum inconsistency of this constraint with the CDM model is solved. They found that this halo is triaxial in the outter skirts with a medium-to-major axis ratio of $0.9$ and minor-to-major axis of $0.8$, which is still very oblate regarding the CDM predictions.\\

However, even when this study solves some inconsistencies with the expected predictions, it demonstrated that small perturbations are important. That is, even when the Sagittarius stream samples the gravitational potential at the outter parts of the halo, where the shape of the inner regions should not be so important, the outter shape is affected to compensate for the change in the regions from inside. This effect takes our attention to a relegated topic: the LMC. In fact, Vera-Ciro et al. demonstrated that the change in shape from the inner regions produces a torque comparable to that of the LMC, which should be taken into account in further researches.\\

%
%Briefly talk about the work of Vera-Ciro2011 and how it is correlated with Vera-Ciro2013.

This section must end emphasizing that we will perform our study based on this previous work on Auriga simulations.

\section{Outline}

Set the outline of the thesis.
