\chapter{Introduction}

\section{About Dark Matter}

In the field of observational astronomy it is very plausible to encounter ourselves with indirect or direct measurements which are not reconcilable with the current understanding of physical theories. Very often, these intriguing measurements are the result of inaccuracy in the measurement device or the asumption of some erroneous or non-precise premises. However, since the early beginings of the 20th century, we have found inconsistencies of observations with the accepted paradigm of physics, which have been reliably proven not to arise from any errors in the premises or devices. These inconsistent measurements principally question the veracity of the Newtionan gravity or even its improved Einstenian version at astronomical and cosmological scales.\\

The principal problem of these inconsistencies can be resumed to the fact that we observe that objects are moving in a certain way that cannot be explained exclusively by the effect of gravitational forces of visible matter. In fact, there is very strong historical evidence that this gravitational effect is heavily \textit{underestimated} if we only take into account the effect of light-interacting matter. To solve this problem without blaming the measurement instruments or the used premises, which have been consistently ruled out as the sources of these discrepancies, there are two important hypothesis to study.\\

One of these hypothesis, which is the most simple to understand, tries to reconcile these observations with Newtonian gravity, or even more, General Relativity, assuming it is completely valid in the cosmological context. Consequently, if there is nothing wrong with the theoretical framework, the premises, nor the instruments, and gravitational effects cannot be explained solely with the matter that we observe, then there must exist some kind of matter that exerts this missing gravitational pull, which we cannot see for some reason. We talk about Dark Matter (DM).\\

Since early as the 1930's, the DM hypothesis has been considered. It is known that Fritz Zwicky proposed the presence of some kind of DM within the Coma cluster of galaxies to explain the cluster attachment despite the galaxie's huge velocities that would have made them scape \cite{Zwicky 1937}. At first, this proposal was not given much attention as there were many other sources of errors to blame, and these observations were not widely replicated. Nevertheless, as years passed, more and more inconsistencies could be explained with the use of this DM hypothesis. It is the case of the famous rotation curves \cite{Faber \& Gallagher 1979,Rubin et al. 1980,Persic et al. 1996} where the ammount of visible matter on galaxies could not account for the centripetal acceleration associated with the tangential velocities of stars within a it. It is, in essence, the argument of Zwicky. Later on, with a better understanding of the space-curving effect of matter predicted by general relativity and the improvement of precision in telescopes, it was posible to measure weak lensing (gravitational distortion of observed objects) through massive clusters of galaxies \cite{Kaiser \& Squires, Wittman et al. 2000, Clowe et al. 2006}. Again, this effect was estimated to be much more dim compared to the actual result of observations, suggesting the existence of DM. This is, in fact, a naturally different approach than those of velocity-curves analysis.\\

Less direct evidence of DM include arguments related to structure formation and the observed Cosmig Microwave Background Radiation (CMBR) \cite{Blumenthal et al. 1984}. On one hand, the structure formation argument is based in the fact that the early stages of the universe were dominated by radiation, which heavily affects visible matter by preventing it to grow clumps from small density anisotropies. Specifically, due to radiation the speed of sound at the early stages of the universe is very close to that of the speed of light, meaning that density anisotropies must be quite large for them to not be dispersed as sound waves. These anisotropies would be evidenced on the CMBR as temperature fluctuations. It has been thoroughly corroborated that these anisotropies are not present on a very precise sampling of the CMBR like the Wilkinson Microwave Anisotropy Probe (WMAP) \cite{WMAP 2003, WMAP 2013} and the Planck Collaboration et al. \cite{Planck Collaboration 2014, PlankPlanck Collaboration 2016}. In the absence of any other kind of pulling force or kind of matter with smaller speed of sound (unaffected by radiation), such as DM, we would not have the same distribution of collapsed structures we observe today (stars, galaxies, clusters) as these density fluctuations would have been dispersed as sound waves. \\


Subsequently, the second important hypothesis to reconcile theory and observations is of oposing nature than that of DM. In this case, we study the posibility that our comprehension of gravity or dynamics is not accurate and is in fact underestimated at cosmological scales. We then talk about theories of modified gravity. In this case of, one of the most successful alteratives of DM are Modified-Newtonian Dynamical (MOND) theories, which modify Newton's inertia \cite{Milgrom 1983}. This is a natural consequence of assuming that the universe is what we observe and there is no such thing as non-visible matter. If we apply this hypothesis to the previously presented missmatches, under a precise tunning of parameters of these new theories, we could obtain the seeked consistency \cite{Sanders \& McGaugh 2002}. Although these theories may succeed in many galactic-regime predictions \cite{Begeman et al. 1991,McGaugh \& de Blok 1998,McGaugh 2012}, they are extremely well constrained for small-scale regimes where they must be accurately Newtonian/Einstenian. This restricts the freedom with which this models may be adjusted to observations and, unlike the DM hypothesis, makes them highly refutable. Some of the biggest problems these theories face are DM remanents that are necessary to completely reconcile predictions and observations \cite{Lokas 2002,McGaugh 2015}. Furthermore, for these theories, the excess of gravitational pull must be centered at the visible-matter distribution, which may not be the case for collisions of galaxies, where it is observed that collisionless DM may decouple from baryonic mass \cite{Clowe et al. 2006}.  \\

For these reasons, DM is the most widely accepted hypothesis to account for these observational discrepancies. Nonetheless, a complete physical picture of DM is still missing and it is one of the biggest puzzles to fully understand the composition of our Universe. Many appealing candidates may be found in the context of Particle Physics, which motivate the existence different kinds of weak-interacting particles through interesting symmetry-seeking theories \cite{Kim 1987, Bertone et al. 2005}.\\

\section{Theoretical background for DM}
In this search for the candidate particle that makes up DM, there have been various proposals from particle physics. One of the most obvious candidates was the neutrino, a lepton of extremely small mass at rest. When physicists studied this option, they discovered that, although it would make sense in principle, there were many problems that suggested that this was not the correct candidate. These arguments come from the analysis of CMB and Structures of Formation and are related to the conclusion that at early stages of the universe, specifically before the decoupling of radiation and matter, neutrinos would be relativistic \cite{Bond et al. 1980,Bertone et al. 2005}, which would result in a non-hierarchical model of formation. Other arguments include constraints on the number of neutrinos surrounding a galaxy \cite{WMAP 2003}. This kind of very-light DM candidates whose thermal energy may significantly affect the proper growth of density anisotropies in the early uiverse, are known as Hot Dark Matter (HDM).\\

If we assume the existence of a heavier particle as a DM candidate, then these particles would not be relativistic before decoupling and therefore it would support a hierarchical model of formation \cite{Blumenthal et al. 1984,Liddle \& Lith 1993}. In this case, we reffer to this particles as Cold Dark Matter (CDM) as their thermal energy is negligible to analyse the collapse of matter. Given its consistency with the observable universe, CDM is the most accepted candidate as a constituyent of DM.\\

Given CDM has negligible thermal energy, it is usually modeled as a set of collisionless particles (minimum cross-section) with gravitational interaction, or, in the continuous case, as an ideal collisionless self-gravitating fluid following the Boltzmann equations \eqref{eq:Boltzmann}. This collisionless feature together with the exclusivity of the gravitational interaction is traduced in a fluid/particle without sense of scale. This means that neither the fluid/particle nor its interaction force possess a well-defined univocal scale parameter. In other words, by a simple rescaling of position and time, we would arrive to a similar system. This characteristic of self-similarity is an important feature of CDM which is widely used on theoretical frameworks as consistent assumptions to simplify the calculations and imply that, at least in a CDM-only universe, many properties of DM structures are self-similar. For instancce, on one hand, Schechter et al. used this argument to analyze the statistics of mass functions in a self-similar universe of DM \cite{Schechter 1976} which has proven to be atoundingly precise for observations. On the other  hand, Bardeen et al. used this argument to analyze the evolution of random Gaussian fluctuations of the DM density field \cite{Bardeen 1986}, resulting in a theoretical framework that is a strong foundation for anyone intending to work on the analysis of DM structures.\\


\begin{equation}
\frac{d\rho}{dt} = \frac{\partial \rho}{\partial t} +\vec{v}\cdot\frac{\partial\rho}{\partial \vec{x}}
-\frac{\partial \Phi}{\partial \vec{x}}\cdot\frac{\partial\rho}{\partial \vec{v}}.
\label{eq:Boltzmann}
\end{equation}

   
By means of example, using a computational approach, Navarro et al. discovered that DM halos (\~ galactic sized clusters of DM) are riged by a universal double-power law \eqref{eq:doublePower} dependent on two parameters, namely $(r_s,\delta_c)$. This result was demonstrated to be independent of the size of the halo or the used cosmology \cite{Navarro et al. 1996, Navarro et al. 1997 ,Power et al. 2003, Navarro et al. 2010}. This is to expect from the self-similarity properties of DM, although it differs from the previously theoretically predicted single-power law  \cite{Gunn \& Gott 1972, White \& Zaritsky 1992}.\\

\begin{equation}
\frac{\rho(r)}{\rho_{crit}} = \frac{\delta_c}{(r/r_s)(1+r/r_s)^2}.
\label{eq:doublePower}
\end{equation}

Now, the collisionless models for CDM may also be applied as a good approximation for star dynamics or to model other cosmological objects with negligible cross-section according to their extreme low densities on the universe, such as Black Holes (BH). By ways of example, we can take the Boltzmann transport equation \eqref{eq:Boltzmann} in a convenient set of coordinates. For instance, we take this collisionless self-gravitating fluid equation in spherical coordinates to analyze the dynamics of stars in axisymmetric galaxies. Now, to get rid of the difficult phase-space density distribution function $f(\vec{x},\vec{v})$ we conveniently integrate the multiplication of the Boltzmann equation with a power of a velocity component, over the velocity space. The result of this integration is the expresion of the collisionless fluid in terms of more observable quantities such as the velocity means $\overline{v_r},\overline{v_z},\overline{v_\phi}$, the velocity correlation matrix $\sigma$ and the density field $\rho$ and the gravitational accelerations $a_r,a_z,a_\phi$. These equations are called Jeans equations in honor of James Jean who was the first to apply this knowledge in the cosmological context \cite{Jeans 1915}. For the case of an axisymmetric stable system \cite{Loebman et al. 2012}, these equations read:  

\begin{align}
a_r & = \sigma^2_{rr} \frac{\partial \ln \nu}{\partial r} + \frac{\partial \sigma^2_{rr} }{\partial r} + \sigma^2_{rz} \frac{\partial \ln \nu}{\partial z} + \frac{\partial \sigma^2_{rz} }{\partial z} +  \frac{\sigma^2_{rr}}{r}-\frac{\sigma^2_{\phi\phi}}{r}-\frac{\overline{v_\phi}^2}{r}\\
a_z & =  \sigma^2_{rz} \frac{\partial \ln \nu}{\partial r} + \frac{\partial \sigma^2_{rz} }{\partial r} + \sigma^2_{zz}\frac{\partial \ln \nu}{\partial z} + \frac{\partial \sigma^2_{zz} }{\partial z} +  \frac{\sigma^2_{rz}}{r},
\end{align}

where the angular acceleration is null for stability reasons. And the the spatial density $\rho$ is encoded in the stellar number density distribution $\nu$.\\

In this case, we modelled stars as a collisionless fluid in the presence of a gravitational potential $\Phi$ which produces the corresponding accelerations. These accelerations may be obtained by the integration of visible matter, but, as expected, it is underestimated. In this case we have a precise theoretical framework to deduce, given the stellar observable properties, what its the discrepancy in accelerations and in this way calculate the missing distribution of DM.\\


\section{Constraining the Milky Way's DM Halo}
So far, DM presence can only be measured through its gravitational effect on the surrounding visible matter. 
One of best the astronomical systems that can be used to probe DM on astronomical scales is our own galaxy: the Milky Way (MW).
Probing the DM density field around our galaxy (it's so-called DM halo) can shed light on the nature of DM \cite{Read \& Moore 2005, Nipoti et al. 2007} and our
galaxy's formation history \cite{Read et al. 2008,Read et al. 2009,Vera-Ciro et al. 2011}.\\

DM haloes have two important features that could be constrained. 
On one hand there is the density profile, which has been demonstrated to follow an approximately universal model \eqref{eq:doublePower} \cite{Navarro et al. 2003}. 
On the other hand, there is the halo shape, which is directly related to its spin. 
According to the hierarchical model of structure formation, due to the anisotropic history of accretion DM haloes are triaxial and therefore, their shape and spin are important characteristics to \textbf{diagnose} their formation history \cite{Bardeen 1986}.\\

In this sense, it is of special interest to constrain the DM halo shape of the only cosmological object of which we have a tridimensional view from inside: our Milky Way. However, this is a very difficult labour given the observational restrictions of observation. Many approaches have been made to constrain the MW's DM shape. One of them is to make use of theoretical models that relate the content of matter of our galaxy with the gravitational potential.\\

For example, Loebman et al. \cite{Loebman et al. 2012} used the axisymmetric Jean's equations \cite{Jeans 1915} that relate the stellar content of our galaxy, with the radial and axial accelerations. 
The observed accelerations cannot be completely explained by visible matter only and DM presence is needed. Loebman et al. estimated that, around 20Kpc, the DM halo must be perfectly oblate with axis ratio of $q_{DM}=0.47 \pm 0.14$ to account for this discrepancy.\\

Nevertheless, the axial symmetry that characterizes this halo is inherited from the use of axisymmetric Jean's equations. 
Although this is a strong asumption, a more general theoretical background is much more difficult to implement given the difficulty to obtain the needed data from observations. 
Even authors aknowledge that "... while it is premature to declare $q_{DM}=0.47 \pm 0.14$ as a robust measurement of the dark matter halo shape, it is encouraging that the simulation is at least qualitatively consistent with SDSS data in so many aspects".
This demonstrates that this field of study is still very young and any calculated constraint may lead us to a better understanding of our MW's DM halo shape.\\

A more common and strong approach is to use the streams of close dwarf galaxies that have been deformed by the gravitational potential of the MW. 
This effect is very important because the torca generated by the anisotropy in our halo is sensible to its parameters and thus, these streams are strong evidence to constrain the shape of our MW's DM halo \cite{Law \& Majewski 2009, Law \& Majewski 2010, Deg \& Lawrence 2013}. 
In fact, it is known that a static axisymmetric halo cannot simultaneously explain all the features of the Sagittarius leading arm \textbf{Law \& Majewski 2009}. \\

In this context Law and Majewski 2010 proposed an analytical model of the MW consisting of a fixed analytical gravitational potential formed by a Miyamoto-Nagai \cite{Miyamoto \& Nagai 1975} disk, a Hernquist spheroid and a logarithmic halo. 
This halo is triaxial and is characterized by its axial ratios and orientation. 
Given all these parameters, the Sagittarius stream was simulated and evolved forward and backwards in time for various choices of the halo parameters. 
The best fit, compared to a detailed study of the observational properties of the Sagittarius stream, was found at a minor/major axis ratio $(c/a)_{\Phi}=0.72$ and intermediate/major axis ratio $(b/a)_{\Phi}=0.99$. The minor axis of this triaxial halo was found to be pointing in some direction contained in the galactic disk plane. \\

This sophisticated model succeeded at simultaneously reproducing the radial velocity and angular position trends of the Sagittarius leading arm, which were troublesome to model with simpler approaches. 
Nevertheless, the coexistence of a triaxial DM halo and an axisymmetric galactic disk is not supported by Cold Dark Matter (CDM) models \cite{Debattista et al. 2008}. 
Specifically, it is expected that the DM and gas distributions are correlated in the sense that matter is accreted similarly as is DM, being the interaction properties the principal difference in the behaviour. 
Having this into account, gas and DM must have aligned angular momenta to certain extent because all kinds of matter are expected to be accreted from the same cosmic structures. 
In other words, it is expected for minor axes to be rather aligned. 
Furthermore, due to stability reasons and a historical interaction, the matter distribution should be non-axisymmetric in the presence of a non-axisymmetric halo potential.\\

Law \& Majewski comment in their paper: "... by no means do they (results) represent best-fit models in a statistical sense. Therefore, the predictions made cannot be considered exclusive or definitive but serve to guide where future observations could focus to distinguish between various models.". 
Particularly, this discrepancy with the current CDM paradigm may be a feature of the specific model. 
Other important observational constraints were dismissed in this study, such as the non-symmetric influence of the Large Magellanic Cloud (LMC). 
This feature may obviate the triaxial halo and produce a more CDM-consistent model. 
However, observationally obtaining the detailed information of the LMC needed for this kind of research is extremely difficult.\\

Studies of this kind are by nature non-conclusive(deterministic?, not so conclusive?) due to the dificulty in obtaining precise information from observations. 
In observations, we take 2-dimensional snapshots of the sky and therefore, we loose resolution of the radial density field due to screening. 
This makes the process of obtaining  a tridimensional view of a cosmological-scale object an extremely difficult endeavour. 
Furthermore, we can determine radial velocities with dopler effect, but there is no obvious way of obtaining tangencial velocities. 
Bearing this in mind, any study which is sensible to very detailed observational parameters for obtaining non-direct measurements of the DM density field will be either non-conclusive for reasonable-difficulty models (as is the process of constraining), or must be extremely sophisticated to achieve a significantly conclusive result.\\

\section{State of the art on MW simulations}
To address this specific difficulty of obtaining information from observations, there is a vast and important field consisting in the modeling of the non-linear behaviour of matter. 
This is with the objective of numerically simulating the universe at a wide range of scales and produce consistent systems of which we have full control of their parameters at all stages. 
In this sense, a computer may become a virtual cosmological-scale laboratory, where we may run an experiment having full control over its initial conditions to compare different outputs and support theoretical frameworks. \\


Cosmological simulations are usually restricted to modelling DM as a non-collisional fluids \eqref{eq:Boltzmann} and gas as an Eulerian collisional fluids \eqref{eq:Euler} whose thermal effects are important.  Efficiently solving these systems of non-linear equations, specially the ones for an Eulerian fluid, is an intricate puzzle and it is still an open and improving field of research. 
Difficulties in the numerical modeling of these fluids arise from numerical instabilities and the wide range of values that quantities take in the context of cosmological objects, which can expand in several orders of magnitude, are no much different than actual field discontinuities which are very difficult to treat in a numerical way. It is clear that these simulations are limited to some resolution depending on the current computing power. This resolution is variable between simulations and is adjustable to the specific objective of the research.\\



In a historical context, numerical astrophysics have experienced a parallel growth with computing power and numerical methods. As we stated earlier, collisionless fluids are not restricted to the modeling of DM but can also be applied to cosmological objects with negligible cross-section. In this sense, as early as 1960's, with the huge advances in computation, it became possible for theoretical astrophysicists to run small-sized simulations of two-dimensional galaxies. This is the case of Miller \& Pendergast \cite{Miller \& Prendergast 1968} and Hohl \& Hockney  \cite{Hohl \& Hockney 1969}, who tried to recover the spiral stable form of galaxies like our MW. At this point in history, the computational power was not sufficient to even attempt a proper solution of collisional fluids. This is a task that requires extreme care and computational power even today. Therefore, to simulate the important termal effects of collision of gas clouds they emulated temperatures as random peculiar velocities of particles and implemented some cooling process in which some particles lose energy \cite{Miller et al. 1970}. With this work, and some insight from Jerry Ostriker and James Peebles \cite{Ostriker \& Peebles 1973} it was demonstrated that galactic disks cannot be stable on their own and need some sort of additional radial pull (DM). In these simulations it was also verified that spiral branches of galaxies are a consequence of the propagation of small density fluctuations driven by disipation of energy by gas clouds (particles). However, it was also demonstrated by Toomre \& Toomre \cite{Toomre \& Toomre 1972} that these anisotropies could also be caused by tidal forces of close encounters between galaxies.\\

\begin{align}
&\frac{d\rho}{dt} + \rho \vec{\nabla}\cdot\vec{v} = 0\\
&\frac{d\vec{v}}{dt} = -\frac{\vec{\nabla}P}{\rho} - \vec{\nabla} \Phi \\
&\frac{du}{dt} = -\frac{P}{\rho}\vec{\nabla}\cdot\vec{v} - \frac{\vec{\Lambda(u,\rho)}}{\rho}\\
& P = (\gamma -1 )\rho u
\label{eq:euler}
\end{align}


Even with the advances of computational power each two years following Moore's law, the numerical methods used to solve simulations quickly became obsolete year by year and needed to be pushed forward to an optimization of the computational resources and numerical resolution. In this way, computational astrophysics evolved from brute-force $N$-body simulations to the use of tree-based simulations \cite{Barnes \& Hut 1986}, to even dynamical-mesh simulations \cite{Berger \& Colella 1989}. This co-evolution between numerical methods and astrophysics, together with the exponential growth of computational power, eventually made possible the performance of Cosmological-sized DM-only simulations such as Millennium wich could reproduce the observed cosmological structures. \\ 

In this context of precise CDM-consistent DM-only simulations, the analysis of gas was usually imprinted through the use of semi-analytic methods that took the output numerical density field and performed analytical calculations to match the distribution of a certain number of visible-matter properties that we observe today \cite{Lemson et al. 2006}. Although these semi-analytical methods did not trace the evolution of gas alongside with DM troughout the simulation, it was the most realistic model that could be performed to do a proper analysis of gas. By this time, numerical methods trying to actually simulate gas suffered from a process of over-cooling in which gas collapsed too quickly and did not produce stable galaxy-sized disk like we observe today \cite{cooling}.\\

This over-cooling process was understood to be a consequence of not having into account events of energy transportation that redistributed energy through the gas stopping this rapid collapse. The principal sources of this energy transport are now known to be supernovae (SN) explosions, radiation from cosmic rays and the Active Galactic Nucleus (AGN). These terms are usually refferred to as stellar feedback, radiation pressure and AGN feedback, respectively. They enclose modern physics processes which cannot be fully simulated due to the computational limitations and must be estimated with general recipies with some free parameters. For instance, in case of SNe explosions this process of energetic feedback is simulated by isotropically liberating some ammount of kinetic energy (radial velocity recoil) as well as some ammount of thermal energy (Temperature), to the surrounding gas cells. \\

 A decade ago, these feedback processes were not as well understood nor well modeled as they are today. For this reason, and the advances in technology, it has been possible only until recently the simulation of an unprecedented set of 30 galactic-sized objects like our MW, tracing the evolution of normal matter alongside with DM with exceptional accuracy. This project is called Auriga, \cite{Auriga} and not only it has state-of-the-art energetic feedback physics which do not need recalibration of parameters in terms of resolution unlike other less consistent approaches, it is run with the novel hydrodynamic code AREPO
\cite{Arepo}. This code combines a moving Voronoi tessellation with the
finite volume approach and in this way, it solves the principal sources of nummerical errors from both important paradigms of computational hydrodynamics in the cosmological context. Morover, it is the first time that a consistent Magentic Field could be simulated in these kind of simulations \cite{parkmorr}.\\
 

\section{Synergy between Theory, Observations and Simulations }
In the CDM paradigm, we have fully theoretical studies \cite{Bardeen et al. 1986,Schechter} principally focused in the analysis of Gaussian random fields and the properties of self-similarity that DM must possess. 
These theoretical frames are then supported by CDM simulations \cite{Cualquier estudio de CDM hace referencia a esos pilares} and, if possible, by observations. 
In fact, these theories are usually thoroughly verified and complemented through simulations, given their convenient malleability, before being directly applied to observations.\\% Redactar mejor (simbiosis entre teor[ia analitica, simulaciones numericas, y observaciones.

One good example of this synergy between analytical models, numerical simulations and observational data is evidenced in the work of Vera-Ciro et al. (2011-2013). In 2011, Vera-Ciro et al. studied the shape of a set of four simulated MW-like DM galaxies with the objective of complementing the predictions of the CDM paradigm. Specifically, the hierarchical model of structure formation predicts that the halo shape is correlated with the environment given that it determines the structures from which it accretes matter \cite{referencias sobre relacion entre forma y entorno}. However, theoretical studies of halo shapes are restricted to the correlations at reshift 0 and do not say much about their history of formation. Intuitively, it is expected that halo shapes vary with the radius taking into account that accretion occurrs at progresively bigger radii in history and that the cosmic structures that determine environments evolve during that time. Due to the collisionless nature of DM, inner shells can interact with outer shells only in a gravitational way. This means that the historical shape is somewhat conserved in the radial shape profile.\\

Vera-Ciro et al.  showed in 2011 that the radial profile of the halo shape is indeed correlated with its accretion history and environment. Furthermore, due to the increase in the cross section of the halos, which contributes to the scattering of particles, at later stages and bigger radii, they become more oblate/spherical. These results helped to obtain more insight about the galactic dynamics of formation and also suggested some guidelines to improve Law \& Majewski 2010 study.\\

In 2013, Vera-Ciro et al. proposed an improved study based on the one by Law \& Majewski in 2011. Vera-Ciro et al. proposed a halo that is perfectly oblate at inner regions and transitions smoothly to a triaxial halo in the outter-skirts. With this, the angular momentum inconsistency of this constraint with the CDM model is solved. They found that this halo is triaxial in the outter skirts with a medium-to-major axis ratio of $0.9$ and minor-to-major axis of $0.8$, which is still very oblate regarding the CDM predictions.\\

However, even when this study solves some inconsistencies with the expected predictions, it demonstrated that small perturbations are important. That is, even when the Sagittarius stream samples the gravitational potential at the outter parts of the halo, where the shape of the inner regions should not be so important, the outter shape is affected to compensate for the change in the regions from inside. This effect takes our attention to a relegated topic: the LMC. In fact, Vera-Ciro et al. demonstrated that the change in shape from the inner regions produces a torque comparable to that of the LMC, which should be taken into account in further researches.\\



%
%Briefly talk about the work of Vera-Ciro2011 and how it is correlated with Vera-Ciro2013.

This section must end emphasizing that we will perform our study based on this previous work on Auriga simulations.

\section{Outline}

Set the outline of the thesis.
