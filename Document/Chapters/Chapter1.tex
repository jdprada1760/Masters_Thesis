\chapter{Introduction}

\section{About Dark Matter}

In observational astronomy it possible to find measurements which are not reconcilable with the current understanding of physical theories. Sometimes, these measurements are the result of instrumental error or incorrect theoretical assumptions. However, since the early beginings of the 20th century, we have found inconsistencies of observations with the accepted paradigm of physics that challenge the veracity of the Newtionan gravity or even its improved Einstenian version at astronomical and cosmological scales.\\

We observe that objects are moving in a certain way that cannot be explained exclusively by the effect of gravitational forces of visible matter. Specifically, observations show that there is an inmense lack of matter to explain the gravitational pull reflected on observed dynamics. To solve this problem there are two important hypothesis to study.\\

One of these hypothesis tries to reconcile these observations with Newtonian/Einsteinian gravity assuming it is completely valid in the cosmological context. Consequently, there must exist some kind of matter that exerts this missing gravitational pull, which we cannot see for some reason. We talk about Dark Matter (DM).\\

Since early as the 1930's, the DM hypothesis has been considered. It is known that pioner Fritz Zwicky proposed its presense within the Coma cluster of galaxies to explain the cluster attachment despite the galactic huge velocities that would have made them scape \cite{Zwicky_1937}. At first, this proposal was not given much attention as there were many other sources of errors to blame and these observations were not widely replicated. Nevertheless, as years passed, more and more inconsistencies could be explained with the use of this DM hypothesis. It is the case of the famous rotation curves \cite{Faber_and_Gallagher_1979,Rubin_et_al._1980,Persic_et_al._1996} where the ammount of visible matter on galaxies could not account for the centripetal acceleration associated with the tangential velocities of stars within a it. Later on, with a better understanding of the space-curving effect of matter predicted by general relativity it was possible to measure weak lensing (gravitational distortion of observed objects) through massive clusters of galaxies \cite{Kaiser_and_Squires,Wittman_et_al._2000,Clowe_et_al._2006}. Again, this effect was estimated to be much dimmer compared to observations.\\

Less direct evidence of DM is related to structure formation and the observed Cosmig Microwave Background Radiation (CMBR) \cite{Blumenthal_et_al._1984}. On one hand, the early stages of the universe were dominated by radiation, which heavily affected visible matter by preventing it to grow clumps (form structures) from small density anisotropies. Specifically, due to radiation, the speed of sound is very close to that of the speed of light, meaning that density perturbations must be large to not be dispersed as sound waves. These anisotropies would be evidenced on the CMBR as temperature fluctuations. It was corroborated that these anisotropies are not present, through precise samplings of the CMBR like the Wilkinson Microwave Anisotropy Probe (WMAP) \cite{WMAP_2003,WMAP_2013} and the Planck Collaboration et al. \cite{Planck_Collaboration_2014,Planck_Collaboration_2016}. In the absence of any other kind of pulling force or kind of matter with smaller speed of sound (unaffected by radiation), we would not have the same distribution of collapsed structures (stars, galaxies, clusters) we observe today. \\


Subsequently, second important hypothesis studies the posibility that our comprehension of gravity or dynamics erroneous at cosmological scales. Then we talk about theories of modified gravity. One of the most successful alteratives of DM are Modified-Newtonian Dynamical (MOND) theories, which modify Newton's inertia law \cite{Milgrom_1983}. If we apply this hypothesis to the previous missmatches, under a precise tunning of parameters, we could obtain the seeked consistency \cite{Sanders_and_McGaugh_2002}. Although these theories may succeed in many galactic-regime predictions \cite{Begeman_et_al._1991,McGaugh_and_de_Blok_1998,McGaugh_2012}, they are extremely well constrained for small-scale regimes by Newtonian/Einsteinian gravity. This restricts the freedom with which this models may be adjusted to observations and makes them highly refutable. Some of the biggest problems these theories face are DM remanents that are necessary to completely reconcile predictions and observations \cite{Lokas_2002,McGaugh_2015}. Furthermore, for these theories, the excess of gravitational pull must be centered at the visible-matter distribution, which may not be the case for collisions of galaxies where it is observed that collisionless DM may decouple from baryonic mass \cite{Clowe_et_al._2006}.\\

For these reasons, DM is the most widely accepted hypothesis to account for observational discrepancies. Nonetheless, a complete physical picture of DM is still missing and it is one of the biggest puzzles to fully understand the composition of our Universe. Many appealing candidates may be found in the context of Particle Physics, which motivate the existence different kinds of weak-interacting particles through interesting symmetric theories \cite{Kim_1987,Bertone_et_al._2005}.\\

\section{Theoretical background for DM}
In the search for the candidate particle that makes up DM, there have been various proposals from particle physics. One of the most obvious candidates was the neutrino, a lepton of extremely small mass at rest. Although neutrinos fulfilled the basic properties of DM, there were many problems when reconciling with observations. These problems come from the analysis of CMB that constrains that at early stages of the universe neutrinos would be relativistic \cite{Bond_et_al._1980,Bertone_et_al._2005}, which would result in a non-hierarchical model of formation. Other problems include constraints on the density of neutrinos \cite{WMAP_2003}, which is not enough to account for missing matter. This kind of very-light DM candidates whose thermal energy significantly affects the proper growth of density anisotropies in the early uiverse, are known as Hot Dark Matter (HDM).\\

If we assume the existence of a heavier particle as a DM candidate, then these particles would not be relativistic at early stages and therefore they would support a hierarchical model of formation \cite{Blumenthal_et_al._1984,Liddle_and_Lith_1993}. In this case, we reffer to this particles as Cold Dark Matter (CDM). Given its consistency with the observable universe, CDM is the most accepted candidate as the principal constituyent of DM.\\

Given that CDM has negligible thermal energy, it is usually modeled as a set of gravitating collisionless particles, or, in the continuous case, as a collisionless self-gravitating Boltzmann fluid \eqref{eq:Boltzmann}. As a consequence, neither the fluid/particle nor its interaction force possess a well-defined scale parameter. In other words, by a simple rescaling of position and time, we would arrive to an equivalent system. This characteristic of self-similarity is widely used on theoretical frameworks to simplify the calculations and imply that, many properties of DM structures are self-similar. For instance, on one hand Schechter used this argument to analyze the statistics of mass functions in a self-similar universe of DM \cite{Schechter_1976} which has a very precise matching with observations. On the other hand, Bardeen used this self-similarity to analyze the evolution of random Gaussian fluctuations of the DM density field \cite{Bardeen_1986}, resulting in a theoretical framework that is a strong foundation for any work on the analysis of DM structures.\\


\begin{equation}
\frac{d\rho}{dt} = \frac{\partial \rho}{\partial t} +\vec{v}\cdot\frac{\partial\rho}{\partial \vec{x}}
-\frac{\partial \Phi}{\partial \vec{x}}\cdot\frac{\partial\rho}{\partial \vec{v}}.
\label{eq:Boltzmann}
\end{equation}

   
By means of example, using a computational approach, Navarro et al. discovered that DM halos (galactic sized clusters of DM) are riged by a universal double-power law \eqref{eq:doublePower}, which was independent of the size of the halo or the used cosmology \cite{Navarro_et_al._1996,Navarro_et_al._1997,Power_et_al._2003,Navarro_et_al._2010}. This is expected from the self-similarity properties of DM, although it differs from the previously theoretically predicted single-power law  \cite{Gunn_and_Gott_1972,White_and_Zaritsky_1992}.\\

\begin{equation}
\frac{\rho(r)}{\rho_{crit}} = \frac{\delta_c}{(r/r_s)(1+r/r_s)^2}.
\label{eq:doublePower}
\end{equation}

Now, collisionless models for CDM are a good approximation to model star dynamics or other cosmological objects with negligible cross-section, such as Black Holes (BH). Let us take, for example, the Boltzmann transport equation \eqref{eq:Boltzmann} in a convenient set of coordinates to analyze the dynamics of stars in axisymmetric galaxies. If we get rid of the phase-space density function $f(\vec{x},\vec{v})$ we can obtain an expresion for stellar dynamics in terms of observable quantities such as the velocity means $\overline{v_r},\overline{v_z},\overline{v_\phi}$, the velocity correlation matrix $\sigma$, the spatial-density field $\rho$ and the gravitational accelerations $a_r,a_z,a_\phi$. These equations are called Jeans equations in honor of James Jean who was the first to apply this knowledge in the cosmological context \cite{Jeans_1915}. For the case of an axisymmetric stable system \cite{Loebman_et_al._2012}, these equations read:  

\begin{align}
\begin{aligned}
a_r & = \sigma^2_{rr} \frac{\partial \ln \nu}{\partial r} + \frac{\partial \sigma^2_{rr} }{\partial r} + \sigma^2_{rz} \frac{\partial \ln \nu}{\partial z} + \frac{\partial \sigma^2_{rz} }{\partial z} +  \frac{\sigma^2_{rr}}{r}-\frac{\sigma^2_{\phi\phi}}{r}-\frac{\overline{v_\phi}^2}{r}\\
a_z & =  \sigma^2_{rz} \frac{\partial \ln \nu}{\partial r} + \frac{\partial \sigma^2_{rz} }{\partial r} + \sigma^2_{zz}\frac{\partial \ln \nu}{\partial z} + \frac{\partial \sigma^2_{zz} }{\partial z} +  \frac{\sigma^2_{rz}}{r},
\end{aligned}
\label{eq:Jeans}
\end{align}

where the angular acceleration is null for stability reasons and the the spatial density $\rho$ is encoded in the stellar number density distribution $\nu$.\\

In this case, the gravitational potential $\Phi$ which produces the corresponding accelerations. These accelerations may be obtained by the integration of visible matter, but, as expected, they are underestimated. In this case we have a precise theoretical framework to deduce the missing distribution of DM from stellar observable properties.\\


\section{Constraining the Milky Way's DM Halo}
So far, DM presence can only be measured through its gravitational effect on the surrounding visible matter. 
One of best the astronomical systems that can be used to probe DM on astronomical scales is our own galaxy: the Milky Way (MW).
Probing the DM density field around our galaxy (it's so-called DM halo) can shed light on the nature of DM \cite{Read_and_Moore_2005,Nipoti_et_al._2007} and our
galaxy's formation history \cite{Read_et_al._2008,Read_et_al._2009,Vera-Ciro_et_al._2011}.\\

DM haloes have two important features that could be constrained. 
On one hand there is the universal density profile on equation \eqref{eq:doublePower} \cite{Navarro_et_al._2010}. 
On the other hand, there is the halo shape, which is directly related to its spin. 
According to the hierarchical model of structure formation, due to the anisotropic history of accretion DM haloes are triaxial and therefore, their shape and spin are important characteristics to \textbf{diagnose} their formation history \cite{Bardeen_1986,Vera-Ciro_and_Helmi_2013}.\\

In this sense, it is of special interest to constrain the DM halo shape of the only cosmological object of which we have a tridimensional view from inside: our Milky Way. However, this is a very difficult labour given the observational restrictions. Many approaches have been made to constrain the MW's DM shape. One of them is to make use of theoretical models that relate the content of matter of our galaxy with the gravitational potential.\\

For example, Loebman et al. \cite{Loebman_et_al._2012} used the axisymmetric Jean's equations on \eqref{eq:Jeans}. 
The observed accelerations cannot be completely explained by visible matter only and DM presence is needed. Loebman et al. estimated that, around 20Kpc, the DM halo must be perfectly oblate with axis ratio of $q_{DM}=0.47 \pm 0.14$ to account for this discrepancy.\\

Nevertheless, this axial symmetry is inherited from the use of axisymmetric Jean's equations. 
Although this is a strong asumption, a more general method is much more difficult to implement given the difficulty to obtain the needed data from observations. 
Even authors aknowledge that:

\blockquote{... while it is premature to declare $q_{DM}=0.47 \pm 0.14$ as a robust measurement of the dark matter halo shape, it is encouraging that the simulation is at least qualitatively consistent with SDSS data in so many aspects.}

This shows that this field of study is still young and any constraint may lead us to a better understanding of our MW's DM halo shape.\\

A more stronger approach is to use the streams of close dwarf galaxies that have been deformed by the gravitational potential of the MW. 
This effect is very important because the torca generated by the anisotropy in our halo is sensible to its shape parameters \cite{Law_and_Majewski_2009,Law_and_Majewski_2010,Deg_and_Lawrence_2013}. 
In fact, it is known that a static axisymmetric halo cannot simultaneously explain all the features of the Sagittarius leading arm \cite{Law_and_Majewski_2009}. \\

In this context Law and Majewski 2010 proposed an analytical model of the MW consisting of a fixed analytical gravitational potential formed by a Miyamoto-Nagai \cite{Miyamoto_and_Nagai_1975} disk, a Hernquist spheroid and a logarithmic halo. 
This halo is triaxial and is characterized by its axial ratios and orientation. 
Given all these parameters, the Sagittarius stream was simulated and evolved forward and backwards in time for various choices of the halo parameters. 
The best fit, compared to a detailed study of the observational properties of the Sagittarius stream, was found at a minor/major axis ratio $(c/a)_{\Phi}=0.72$ and intermediate/major axis ratio $(b/a)_{\Phi}=0.99$. The minor axis of this triaxial halo was found to be pointing in some direction contained in the galactic disk plane. \\

This sophisticated model succeeded at simultaneously reproducing the radial velocity and angular position trends of the Sagittarius leading arm, which were troublesome to model with simpler approaches. 
Nevertheless, the coexistence of a triaxial DM halo and an axisymmetric galactic disk is not supported by Cold Dark Matter (CDM) models \cite{Debattista_et_al._2008}. 
Specifically, it is expected that the DM and gas distributions are correlated in the sense that matter is accreted from the same cosmic structures as is DM. 
Therefore, gas and DM must have aligned angular momenta to certain extent, i.e. for minor axes must be aligned. 
Furthermore, to guarantee stability reasons and a historical interaction, DM distributions should be axisymmetric in the presence of an axisymmetric disk potential \cite{Ostriker_and_Peebles_1973}.\\

Law and Majewski comment in their paper: 

\blockquote{... by no means do they (results) represent best-fit models in a statistical sense. Therefore, the predictions made cannot be considered exclusive or definitive but serve to guide where future observations could focus to distinguish between various models.}
 
Particularly, this discrepancy with the current CDM paradigm may be a feature of the specific model. 
Other important observational constraints were dismissed in this study, such as the non-symmetric influence of the Large Magellanic Cloud (LMC). 
This feature may obviate the triaxial halo and produce a more CDM-consistent model. 
However, observationally obtaining the detailed information of the LMC needed for this kind of research is extremely difficult.\\

Studies of this kind are by nature non-exact due to the dificulty in obtaining precise information from observations. 
In observations, we take 2-dimensional snapshots of the sky and loose resolution of the radial density field due to screening of matter. 
This makes the process of obtaining  a tridimensional view of a cosmological-scale object an extremely difficult endeavour. 
Furthermore, we can determine radial velocities with dopler effect, but there is no obvious way of obtaining tangencial velocities. 
Bearing this in mind, any study which is sensible to detailed observational parameters for obtaining non-direct measurements of the DM density field, will be either non-conclusive for reasonable-difficulty models, or must be extremely sophisticated to achieve a significantly exact result.\\

\section{State of the art on MW simulations}
To address the observational difficulties, there is an important field consisting in the modeling of the non-linear behaviour of matter. 
This is with the objective of numerically simulating the universe at a wide range of scales and produce systems of which we have full control of their parameters. 
In this sense, a computer may become a virtual cosmological laboratory, where we may run different experiment having control over their initial conditions, and in this way support or verify theoretical frameworks. \\


Cosmological simulations are usually restricted to modelling DM as a non-collisional fluids \eqref{eq:Boltzmann} and gas as an Eulerian collisional fluid \eqref{eq:Euler}.  Efficiently solving these systems of non-linear equations, is an intricate puzzle and it is still an open and improving field of research. 
Difficulties in the modeling of these fluids arise from numerical instabilities and the wide range of values that quantities take in the cosmological context, which may quickly expand in several orders of magnitude, becoming problematic numerical discontinuities. Depending on the current computing power, these simulations are limited to some resolution, which is adjustable to the specific objective of the research.\\


\begin{align}
&\frac{d\rho}{dt} + \rho \vec{\nabla}\cdot\vec{v} = 0\\
&\frac{d\vec{v}}{dt} = -\frac{\vec{\nabla}P}{\rho} - \vec{\nabla} \Phi \\
&\frac{du}{dt} = -\frac{P}{\rho}\vec{\nabla}\cdot\vec{v} - \frac{\vec{\Lambda(u,\rho)}}{\rho}\\
& P = (\gamma -1 )\rho u
\label{eq:Euler}
\end{align}


In a historical context, numerical astrophysics have experienced a parallel growth with computing power and numerical methods. As we stated earlier, collisionless fluids are not restricted to the modeling of DM but can also be applied to cosmological objects with negligible cross-section. In this sense, as early as 1960's, it became possible for theoretical astrophysicists to run small-sized simulations of two-dimensional galaxies. This is the case of Miller and Pendergast \cite{Miller_and_Prendergast_1968} and Hohl and Hockney  \cite{Hohl_and_Hockney_1969}, who tried to recover the spiral stable form of galaxies like our MW. At this point, the computational power was not sufficient to even attempt a proper solution of collisional fluids. Even today, this requires extreme care and computational power. Consequently, to simulate dissipation effects of collision of gas clouds they emulated temperatures as random peculiar velocities of particles and implemented some cooling process in which some particles lose energy \cite{Miller_et_al._1970}. With this work, and some insight from Jerry Ostriker and James Peebles \cite{Ostriker_and_Peebles_1973} it was demonstrated that galactic disks cannot be stable on their own and need some sort of additional radial pull (DM). In these simulations it was also verified that spiral branches of galaxies are a consequence of the propagation of small density fluctuations driven by disipation of energy by gas clouds. Parallel to this work, it was also demonstrated by Toomre and Toomre \cite{Toomre_and_Toomre_1972} that these disruptions could also be caused by tidal forces of close encounters between galaxies.\\


Even with the advances of computational power following Moore's law, the numerical methods used to solve simulations quickly became obsolete year by year and needed to be pushed forward to an optimization of the computational resources. In this way, computational astrophysics evolved from brute-force $N$-body simulations to the use of tree-based simulations \cite{Barnes_and_Hut_1986}, to even dynamical-mesh simulations \cite{Berger_and_Colella_1989}. This co-evolution between numerical methods and astrophysics together with the exponential growth of computational power, eventually made possible the performance of Cosmological-sized DM-only simulations such as Millennium wich could reproduce the observed cosmological structures at a wide range of scales. \\ 

In this context, the analysis of gas was usually imprinted through the use of semi-analytic methods. They took the numerical output and performed analytical calculations to match the distribution of a certain number of visible-matter properties that we observe today \cite{Lemson_et_al._2006,De_Lucia_et_al._2006}. Although these semi-analytical methods did not trace the evolution of gas alongside with DM, it was the most realistic gas model that could be performed. By this time, numerical methods trying to actually simulate gas suffered from a process of over-cooling in which gas collapsed too quickly and did not produce stable galaxy-sized disk like we observe today \cite{Katz_and_Gunn_1991}.\\

This over-cooling process was understood to be a consequence of not having into account events of energy redistribution that stopped the rapid collapse. The principal sources of this energy transport are now known to be supernovae (SN) explosions, radiation from cosmic rays and the Active Galactic Nucleus (AGN). These terms are usually refferred to as stellar feedback, radiation pressure and AGN feedback, respectively. They enclose modern physics processes which cannot be fully simulated and must be estimated with general recipies with some free parameters. For instance, in case of SNe explosions this process of energetic feedback is simulated by isotropically liberating some ammount (free parameter) of kinetic energy (radial velocity recoil) as well as some ammount of thermal energy (Temperature), to the surrounding gas cells. \\

 A decade ago, these feedback processes were not as well understood as they are today. For this reason, it has been possible only until recently the simulation of an unprecedented set of 30 galactic-sized objects like our MW, tracing the evolution of normal matter alongside with DM with exceptional accuracy. This project is called Auriga, \cite{auriga} and not only it has state-of-the-art energetic feedback physics, it is run with the novel hydrodynamic code AREPO
\cite{arepo}. This code combines a moving Voronoi tessellation with the
finite volume approach and in this way, it solves the principal sources of nummerical errors from both important paradigms of computational hydrodynamics in the cosmological context. Morover, it is the first time that a consistent Magentic Field could be simulated in these kind of simulations \cite{Pakmor_et_al._2017}.\\
 

\section{Synergy between Theory, Observations and Simulations }
In the CDM paradigm, we have fully theoretical studies \cite{Bardeen_1986,Schechter_1976} principally focused in the analysis of Gaussian random fields and the properties of self-similarity that DM must possess. 
These theoretical frames are then supported by CDM simulations and, if possible, by observations. 
In fact, these theories are usually thoroughly verified and complemented through simulations, given their convenient malleability, before being directly applied to observations.\\% Redactar mejor (simbiosis entre teor[ia analitica, simulaciones numericas, y observaciones.

One good example of this synergy is evidenced in the work of Vera-Ciro et al. (2011-2013). In 2011, Vera-Ciro et al. studied the shape of a set of four MW-like DM-only galaxies from Aquarius simulations \cite{aquarius}, with the objective of complementing the predictions of the CDM paradigm. Specifically, the hierarchical model of structure formation predicted that the halo shape must be correlated with the environment \cite{Tormen_et_al._1997,Colberg_et_al._1999}. However, theoretical studies of halo shapes are restricted to the correlations at reshift 0 and do not say much about their history of formation. Intuitively, it is expected that halo shapes vary with the radius taking into account that accretion occurrs at progresively bigger radii in history and that the cosmic structures that determine environments evolve during that time. Due to the collisionless nature of DM, inner shells of the halo can interact with outer shells only in a gravitational way. This means that the historical shape must be conserved in the radial shape profile.\\

Vera-Ciro et al.  showed in 2011 that the radial profile of the halo shape is indeed correlated with its accretion history and environment. Furthermore, due to the increase in the cross section of the halos, which contributes to the scattering of particles, at later stages and bigger radii, they become more oblate/spherical. These results helped to obtain more insight about the galactic dynamics of formation and also suggested some guidelines to improve Law and Majewski 2010 study.\\

In 2013, Vera-Ciro and Helmi proposed an improved study based on the one performed by Law and Majewski in 2010. They modeled a halo as perfectly oblate at inner regions, which transitions smoothly to a triaxial halo in the outter-skirts. With this, the angular momentum inconsistency of this constraint with the CDM model is solved. They found that this halo is triaxial in the outter skirts with a medium-to-major axis ratio of $0.9$ and minor-to-major axis of $0.8$, which is still very oblate regarding the CDM predictions.\\

However, besides solving some inconsistencies with the expected predictions, it demonstrated that small perturbations are important. Specifically, even when the Sagittarius stream samples the gravitational potential at the outter parts of the halo, where the shape of the inner regions should not be important, the outter shape is affected to compensate for the change in the regions from inside. This effect takes our attention to a relegated topic: the LMC. In fact, Vera-Ciro et al. demonstrated that the change in shape from the inner regions produces a torque comparable to that of the LMC, which should be taken into account in further researches.\\

This study comes from DM-only simulations and succeeded at improving previous constraints on the shape of our MW halo from a better understanding of DM. Taking into account the new and improved Auriga simulations, which can model gas alongside DM, our principal motivation in this work is to use these simulations validate previous knowledge about DM, as well as obtaining new insights on the DM-gas relation. Specifically, we want to verify how the shape of a DM halo is affected by radius, redshift and presence of gas.\\

%
%Briefly talk about the work of Vera-Ciro2011 and how it is correlated with Vera-Ciro2013.

\section{Outline}
This thesis is organized as follows: in Chapter 2 we present details about the Auriga simulations in which we perform our calculations, as well as the specifics of the shape-calculating method that we use for our purposes. In Chapter 3 we present our results. And Finally, in Chapter 3 we resume and discuss our work and its future projection.  
