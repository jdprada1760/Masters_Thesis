\chapter{Our results}

In this chapter we present the principal results of our work as well as some CDM-consistent phenomenology. First, we address convergence issues. Secondly, we study the relation of the axial ratios in terms of the radius. On third place, we show our results on the effect of baryons on the moulding of the shape. Finally, we study the evolution of the radial profile of the axial ratios. 

\section{Convergence Analysis}
One of the principal factors that may bias our study is the resolution of the simulations. Fortunately, Auriga simulations have level 3 and level 4 versions of 6 of the gaxies which we can use to analyze the numerical convergence of the results. Resolution may also affect our procedure for calculating halo shapes through the reduction of particles taken into account to calculate the inertia tensor.\\

To illustrate this, in figures \ref{fig:goodConvergence} we compare the halo shape at redshift 0 for level 3 and level 4 simulations. In this case, we can say that there is good convergence of the studied quantities with very small numerical bias. However, this is not the case for every simulated halo.\\

 By way of example, in figures \ref{fig:badConvergence} we present a halo where resolution significantly affected the shape. In this case, although the difference is not extreme, it requires attention and a more careful analysis.\\

\begin{figure}[!ht]
  \centering
  \subfloat[halo 6 DM]{\includegraphics[width=0.7\columnwidth]{./pics/Convergence/halo6_DM_3Vs4_good.png}}
  \hfill
  \subfloat[halo 27 MHD]{\includegraphics[width=0.7\columnwidth]{./pics/Convergence/halo27_MHD_3Vs4_good.png}}
  \caption{Level 3 (green) \& 4 (pink) radial profiles of axial ratios for halo 6 (DM) \& 21 (MHD). Here, it is clear that there is good agreement on the calculated quantities for both levels of resolution.}
  \label{fig:goodConvergence}
\end{figure}



\begin{figure}[!ht]
  \centering
  \subfloat[halo 16 DM]{\includegraphics[width=0.8\textwidth]{./pics/Convergence/halo21_DM_3Vs4_bad.png}}
  \hfill%\hspace{-0.5em}
  \subfloat[halo 23 MHD]{\includegraphics[width=0.8\textwidth]{./pics/Convergence/halo23_MHD_3Vs4_bad.png}}
  \caption{Level 3 (green) \& 4 (pink) radial profiles of axial ratios for halo 16 (DM) \& 23 (MHD). Here, axial ratios show slow convergence. }
  \label{fig:badConvergence}
\end{figure}


For instance, by simple inspection, we notice that there is no apparent systematic way in which resolution affects the halo shape. That is, sometimes the highly-resoluted halo appears rounder and other times it seems more triaxial. This is important for our study as we focus on the analysis of the triaxial properties of the halo. Incidentally, DM-only halos remain unchanged with the exemption of the radial regimes where the number of particles affects our shape-calculating method. However, for MHD simulations, the resolution of gas has a global influence over the axial ratios. We suspect this is caused by continuous exposition of particles to the resolution-sensible baryonic potential. Nevertheless, further calculations are needed to confirm the cause of these discrepancies.\\

Consequently, to rule out our shape method as a cause of these resolution differences, we decide to isolate the few-particle effect on our calculations. To do this, we randomly select DM particles from level 3 halos at $z=0$ to produce 10 samples of approximately the same size as level 4 simulations. We then calculate this few-particle effect, which we show in figures \ref{fig:convergence}. For each radius, we calculate the standard deviation of the sample shape and illustrate a 3-sigma range around the level 3 curve to compare with the respective level 4 shape values.\\  

From graphics on \ref{fig:convergence}, it is clear that the fractional difference is not actually significant and remains under $1\%$ for the majority of the radial profile. It becomes important for radii less than $1Kpc$ due to the lack of particles for approximating an elliptical shape. This is corroborated by the $3\sigma$ range, which also becomes evident around $1Kpc$. We deduce from this analysis that for radii bigger than $1Kpc$, the differences of level 3 and level 4 ellipses cannot be explained as an effect of the lack of particles. This is a confirmation that all kinds of matter are directly affected by resolution due to precision-sensitive events on the history of formation or because the numerical gravitational potential of matter continuously influences surrounding structures. Either way, even for the most resolution-biased cases, we can say that for the purposes of this study, convergence is achieved to a reasonable extent.\\  

\begin{figure}[!ht]
  \centering
  \subfloat[halo 24 MHD]{\includegraphics[width=0.7\columnwidth]{./pics/Convergence/rand_conv_halo24_DM.png}}
  \hfill
  \subfloat[halo 24 DM]{\includegraphics[width=0.7\columnwidth]{./pics/Convergence/rand_conv_halo24_MHD.png}}
  \caption{The few-particle effect on the axial ratios convergence for halo 24 (DM \& MHD). Here level4 curves (magenta) are compared to  the 3$\sigma$ range (clear blue) of the random-sampled curves from level3 (solid blue). We deduce from the fractional difference (green) that discrepancies at $r>>1$Kpc cannot be explained solely with the few-particle bias. }
  \label{fig:convergence}  
\end{figure}

\section{The radial dependence of axial ratios}
One of the our first results is related to the evolution of the DM halo shape in terms of the radius. We expect from previous work that the shape does not remain constant along the radius \cite{Vera-Ciro_et_al._2011}. Specifically, after some time, halos are gradually constructed from inner shells to outer shells through constant accretion of matter from cosmic structures \cite{Tormen_et_al._1997,Tormen_et_al._1998}. Inner shells tend to conserve their shape as a consequence of being shielded from the outside by Gauss law. Outer shells, on the other side, are continuously affected by the gravitational potential from the inside, which makes them prone to shifting orbits. This inner gravitational potential has a \textit{rounding} effect on the outerskirts. For this reason, we expect that halos are more triaxial on inner regions and more spherical at bigger radii. This effect has been corroborated on multiple cosmological simulations \cite{Frenk_et_al._1988,Dubinski_and_Carlberg_1991,Warren_et_al._1992,Cole_and_Lacey_1996,Hayashi_et_al._2007,Bett_et_al._2007,Vera-Ciro_et_al._2011}. \\


\begin{figure}[!ht]
  \centering
  \subfloat[halo 27 DM shape at small radius]{\includegraphics[width=0.5\columnwidth]{./pics/MHD_Vs_DM/level4_DM_halo_27_inner.png}}
  \hfill
  \subfloat[halo 27 DM shape at big radius]{\includegraphics[width=0.5\columnwidth]{./pics/MHD_Vs_DM/level4_DM_halo_27_outter.png}}
  \hfill
  \subfloat[halo 27 MHD shape at small radius]{\includegraphics[width=0.5\columnwidth]{./pics/MHD_Vs_DM/level4_MHD_halo_27_inner.png}}
  \hfill
  \subfloat[halo 27 MHD shape at big radius]{\includegraphics[width=0.5\columnwidth]{./pics/MHD_Vs_DM/level4_MHD_halo_27_outter.png}}
  \caption{DM density for inner (left) and outer (right) parts of the halo 27. We present both versions: DM (up) \& MHD (down). The horizontal and vertical axes are aligned to the major and medium semi-axes respectively. Here, it is evident that this halo is more spherical at bigger radii and more triaxial at the central parts. }
  \label{fig:slices}
\end{figure}

In figures \ref{fig:slices}, we present a halo in which this rounding effect is evident for both degrees of realism. However, to eliminate any possible cualitative bias, on figure \ref{fig:DM_MHD} we present a quantitative version of this effect. There, we include all axial ratios, which clearly become closer to 1 for bigger radii. Additionally, we include a quantification of the triaxiality, namely $T=\frac{1-b/a}{1-c/a}$.\\

 This measurement $T$ tends towards unity when the medium-to-major axis ratio becomes equal to the minor-to-major ratio, i.e. when the halo becomes prolate. When the medium axis is close to the major axis but not to the minor axis, $T$ tends to a null value, i.e. when the halo tends to an oblate shape. In these terms, halos are expected to be more prolate on the inside and more oblate on the outside. Even though the perfect spherical shape has a divergent/undefined $T$ value, prolate shapes are associated with triaxial characterizations and oblate shapes are identified as approximately spherical shapes. However, this can be confirmed on the triaxial $c/a$ Vs $b/a$ plane where we also demonstrate that this is in fact a global tendency for all halos.\\


\begin{figure}
\centering
{\includegraphics[width=1\columnwidth]{./pics/MHD_Vs_DM/level4_halo_27_DM_Vs_MHD.png}}
\caption{Radial profile for axial ratios and triaxiality parameter $T=\frac{1-b/a}{1-c/a}$ from halo 27. This halo has a clear radial tendence towards sphericity (for bigger radii), which can be confirmed with the triaxiality parameter. }
\label{fig:DM_MHD}
\end{figure} 


In figure \ref{fig:Triaxiality_Inner_Outer}, we show the axial ratios on the plane $c/a Vs b/a$. There, each dot labeled by radius represents a specific shape. In this plane, oblate halos are represented by the vertical line $x = 1$, prolate halos are identified on the identity line and spheres are exactly the point $(1,1)$. This gives us a broader idea of the evolution of the shape. The tendency is clear for halos to get rounder with increasing radius. In fact, the difference in shape clear enough that it is possible to identify groups in case the radius label is lost.\\

\begin{figure}[!ht]
  \centering
  \subfloat[Level4 MHD Vs DM at inner regions]{\includegraphics[width=0.7\columnwidth]{./pics/Triaxial_Plane/Triaxiality_Inner_lvl4.png}}
  \hfill
  \subfloat[Level4 MHD Vs DM at outter regions]{\includegraphics[width=0.7\columnwidth]{./pics/Triaxial_Plane/Triaxiality_Outter_lvl4.png}}
  \hfill
  \caption{Axial ratios as shown on $c/a$ Vs $b/a$. Each dot represents a halo shape at some radius. Some observational constraints are plotted alongside our results. Here, dots are clustered, proving the general tendence of halos to get rounder on the outer parts. }
  \label{fig:Triaxiality_Inner_Outer}
\end{figure}


\subsection{The effect of gas on the halo shape}
We have simultaneously corroborated the rounding effect of radius on the halo shape from DM-only and MHD simulations. However, from the parallel presentation of results from MHD and DM simulations, it is noticeable that MHD halos are in general more spherical than DM halos, which is to be expected \cite{Barnes_and_Hernquist_1996,Springel_et_al._2004,Bryan_et_al._2013}.\\

 Unlike DM, gas collapses and generates disks which are much denser than the DM structures. This amplifies the effect of the gravitational potential and, if we apply the same logic, we would expect that the inner regions of the halo are more spherical where there is presence of gas. We expect the same for outter regions but this effect is predicted to be more significant due to a continuous effect of the gravitational potential of inner shells.\\

For instance, recurring again to the figures \ref{fig:slices}, now comparing the graphics vertically, the rounding effect of visible matter is clear. For a more cuantitative illustration of this, we can reffer to figure \ref{fig:DM_MHD}. \\
\begin{figure}[!ht]
  \centering
  \subfloat[Level4 DM inner Vs outer regions]{\includegraphics[width=0.7\columnwidth]{./pics/Triaxial_Plane/Triaxiality_DM_lvl4.png}}
  \hfill
  \subfloat[Level4 MHD inner Vs outter regions]{\includegraphics[width=0.7\columnwidth]{./pics/Triaxial_Plane/Triaxiality_MHD_lvl4.png}}
  \hfill
  \caption{General tendence on the triaxial plane $c/a$ Vs $b/a$. Some observational constraints are plotted alongside our results}
  \label{fig:Triaxiality_DM_MHD}
\end{figure}

Although from previous pictures it is evident that the presence of gas affects the halo shape making it rounder, it is not clear that this effect is amplified for bigger radii. To confirm this, we reccurr again to triaxiality plane on \ref{fig:Triaxiality_DM_MHD}, where this tendency becomes evident.\\


So far, our results are in accordance with previous work. Nonetheless, in the specific case of MW-like galaxy simulations, we have confirmed the expected tendence in an unprecedented statistically significant sample of 30 galaxies form Auriga, compared to the 4-sample galaxies from the previous state-of-the-art Aquarius simulations. Moreover, we confirmed that these results are sustained for the specific case of novel MHD MW-like galaxy simulations where we could analyze effect of gas.\\

\section{Historical shape}
Taking into account the previous fenomenology for halo formation, it is possible to extend 
its reach for the analysis of the historical evolution of the halo shape.\\

 Recalling that inner shells of the halo are isolated from the gravitational effect of outer shells, the only significant source of disruption in time of this radial regime are external structures that perform some torque on them. Outter shells must feel this source of deformation too in addition to the effect from the inner gravitational potential. Consequently, we expect a systematic change on the halo shape with time, which becomes more significant for bigger radii.\\
 
Major events like mergers, may completely disturb a galaxy shapes and erase any memory of it. However, from $z~1$ onwards, these events are very rare \cite{Tormen_et_al._1998} and we expect that any source of disruption is weak and is reduced to the previously mentioned factors. These sources of potential asymmetries influence DM particles towards more spherical orbits \cite{Debattista_et_al._2008}. \\  

\begin{figure}[!ht]
  \centering
  \subfloat[halo 16 DM]{\includegraphics[width=0.7\columnwidth]{./pics/Redshift/halo_16_level3_DM_Z.png}}
  \hfill
  \subfloat[halo 16 MHD]{\includegraphics[width=0.7\columnwidth]{./pics/Redshift/halo_16_level3_MHD_Z.png}}
  \caption{Radial profile (comoving) of axial ratios for halo 16 in terms of redshift (color). This halo maintains its shape until $z\approx 1$ obviating the systematic rounding effect in time from asymmetric potentials. }
  \label{fig:RedshiftGood}
\end{figure}

In figures \ref{fig:RedshiftGood} we present the evolution of the radial profile of the shape of a halo that managed to conserve its integrity until $z \approx 1$. In this case, we show our results in terms of the comoving coordinates to obviate the scale factor and make these profiles comparable. The halo becomes systematically more spherical as it evolves in time, being this effect more relevant for $r>50Kpc$.\\

In figures \label{fig:RedshiftDMbad} we present a special case of a halo that was perturbed at some time around $z\approx 0.5$. It is specially evident because of the discontinuity caused in the radial profile and the large differences in the virial radii. In figure \ref{fig:RedshiftSnaps}, we confirm that the source of this disruption is a moderate sized infalling subhalo around $z\approx 0.5$.  \\  


\begin{figure}[!ht]
  \centering
  \subfloat[halo 21 DM]{\includegraphics[width=0.7\columnwidth]{./pics/Redshift/halo_21_level3_DM_Z.png}}
  \hfill
  \subfloat[halo 21 MHD]{\includegraphics[width=0.7\columnwidth]{./pics/Redshift/halo_21_level3_MHD_Z.png}}
  \caption{Radial profile (comoving) of axial ratios for halo 21 in terms of redshift (color). This halo is disrupted around $z \approx 0.5$ which results in a certain loss of its shape memory.}
  \label{fig:RedshiftBad}
\end{figure}

Now, these results compare radial profiles in comoving coordinates, but in real life we have physical coordinates. In this order of ideas, we can state this preservation of the shape (obviating the rounding effect) without recurring to comoving comparisons.\\

In this case, consider a physical radius $R$ that is well-defined for each redshit. It is the radius at which we are going to perform our historical measurements. For practical purposes let us take, for example, the virial radius at each redshift. Then, as halos are continuosly accreting matter, the virial radius will be in general smaller (in comoving, just as means of comparison) for higher redshifts. This means we are effectively sampling the shape for smaller radii at higher redshifts. Taking into account that the halo shape is well conserved in time, we expect that the historical profile of the axial ratios at a certain physical radius is correlated to the radial profile of the same halo at the present time. \\

To illustrate this, in figures \ref{fig:RedshiftDMTriax16,fig:RedshiftDMTriax21} we present the historical and radial profiles of the previously analyzed halo shapes. For the halo that maintained a consistent shape during time, there is a clear correlation between the historical and radial profiles, both clearly tending to more spherical shapes at lower redshifts and bigger radii. In the case of the halo that had a major disrupting event, this correlation is not clear as an evidence of memory loss by the massive infalling material.\\  

%\begin{verbatim}
\begin{figure}[!ht]
  \centering
  \subfloat[halo 16 MHD]{\includegraphics[width=0.7\columnwidth]{./pics/Redshift/halo_16_level3_MHD_Z_Triax.png}\label{fig:RedshiftDMTriax16}}
  \hfill
  \subfloat[halo 21 MHD]{\includegraphics[width=0.7\columnwidth]{./pics/Redshift/halo_21_level3_MHD_Z_Triax.png}\label{fig:RedshiftDMTriax21}}
  \caption{Historic shape (color dots) Vs actual radial shape (solid black line) on the Triaxiality plane. Each colored dot represents a calculated shape at R Mean 200, for each redshift. It is clear that halos with memory, unlike disrupted halos, have correlation between their historical and radial profiles.}
\end{figure}
%\end{verbatim}







 
