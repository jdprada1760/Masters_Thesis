\chapter{Our results}

In this chapter we are going to present our results. First some remarks about resolution and convergence of the shape within Auriga simulations for DM and MHD. Then we study the radial and historic profiles of DM and MHD halos, where we obtain the expected tendence found by Vera-Ciro et al. 2011. Then present our principal results, which are the ones referring the comparison DM-MHD.

\section{Analysis of convergence}
One of the principal factors that may bias our study is the resolution of the simulations we work with. This is an indicator principally thought to analyze the numerical convergence of the methods used to solve the non-linear equations for matter, and give validity over the output results overall. However, resolution may also directly affect our procedure for calculating halo shapes through the reduction of particles taken into account to calculate the inertia tensor.\\

To illustrate this, in figures \ref{fig:goodConvergenceDM,fig:goodConvergenceMHD} we compare the obtained halo shape at redshift 0 on level 3 and level 4 simulations for a halo in which resolution does not noticeably affect the results. In this case, we can say that there is good convergence of the studied quantities with very small numerical bias. However, this is not the case every simulated halo as they do not evolve similarly and some resolution-sensitive events may influence their history of formation. By way of example, in figures \ref{fig:badConvergenceDM,fig:badConvergenceMHD} we present one of the halos where resolution played the most appreciable role affecting the shape. In this case, although the difference is not extreme, it requires attention and a more careful analysis.\\

\begin{figure}[!ht]
  \centering
  \subfloat[halo 6 MHD]{\includegraphics[width=0.5\columnwidth]{./pics/Convergence/halo6_DM_3Vs4_good.png}\label{fig:goodConvergenceDM}}
  \hfill
  \subfloat[halo 27 DM]{\includegraphics[width=0.5\columnwidth]{./pics/Convergence/halo27_MHD_3Vs4_good.png}\label{fig:goodConvergenceMHD}}
  \caption{Examples of halos where resolution did not have an appreciable effect on the halo shape. Level 3 (blue) and level 4 (red) calculations are in good agreement. \textbf{put labels, bigger axes font, add title} }
\end{figure}



\begin{figure}[!ht]
  \centering
  \subfloat[halo 21 MHD]{\includegraphics[width=0.5\columnwidth]{./pics/Convergence/halo21_DM_3Vs4_bad.png}\label{fig:badConvergenceDM}}
  \hfill
  \subfloat[halo 23 DM]{\includegraphics[width=0.5\columnwidth]{./pics/Convergence/halo23_MHD_3Vs4_bad.png}\label{fig:badConvergenceMHD}}
  \caption{Examples of halos where resolution had an appreciable effect on the halo shape. There is an appreciable difference between level 3 (blue) and level 4 (red) calculations. However it does not modify the halo's shape in any specific way. \textbf{put labels, bigger axes font, add title} }
\end{figure}


For instance, by simple inspection, we notice that there is no apparent systematic way in which resolution affects the halo shape. That is, sometimes the halo appears rounder and some times it is affected towards a more triaxial shape. This is important for our study as we focus our efforts on the analysis of the halo triaxial properties. Incidentally, the DM-only simulated halo shapes remain in general unchanged with the exemption of the radial regimes where the number of particles naturally affects our shape-calculating method. However, for MHD simulations, the resolution of gas influences the measurement not only in the inner parts, where discretization issues are evident, but it has a more global effect. We suspect this is caused by the scattering of particles due to dense structures formed by gas, whose effect is affected by resolution. Nevertheless, further calculations need to be performed to confirm if these resolution biases are directly caused by AllGood's method for calculating shapes, or are caused because structures are indeed affected by numerical errors from the solution of fluid equations of matter.\\

Consequently, we decided to isolate the few-particle effect on our shape calculations without recurring to the less-resoluted simulations of level 4. Taking into account that the resolution difference between level 3 and level 4 simulations is a factor of 8, we randomly selected particles from level 3 halos at redshift 0 to produce 10 samples of approximately the same size as level 4 simulations. We then proceeded to analyze the effect of lowering the number of particles on the calculated shape of the halo. In figures \ref{fig:convergenceMHD,fig:convergenceDM} we ploted the original level 3 shapes as well as the 10 level 3 samples. For each radius, we calculated the standard deviation of the sample shape and illustrated 3-sigma range to compare with the respective level 4 shape values.\\  

From the graphics on \ref{fig:convergenceMHD,fig:convergenceDM}, it is clear that the plotted fractional difference is not actually big and remains under $1\%$ for the majority of the radial profile. It becomes important for radii less than $1Kpc$ due to lack of particles to obtain a good approximation of the elliptical shape. This is corroborated by the $3\sigma$ range, which becomes evident around $1Kpc$. The main conclusion of this convergence analysis is that for radii bigger than $1Kpc$, the differences of level 3 and level 4 ellipses cannot be explained as an effect of the lack of particles. This is a confirmation that all kinds of matter are directly affected by resolution due to precision-sensitive events on the history of formation or because the gravitational potential of matter itself is affected and continuously influences surrounding matter. Either way, even for the most resolution-biased cases, we can say that for the purposes of this study, convergence is achieved to a reasonable extent.\\  

\begin{figure}[!ht]
  \centering
  \subfloat[halo 24 MHD]{\includegraphics[width=0.5\columnwidth]{./pics/Convergence/rand_conv_halo24_DM.png}\label{fig:convergenceMHD}}
  \hfill
  \subfloat[halo 24 DM]{\includegraphics[width=0.5\columnwidth]{./pics/Convergence/rand_conv_halo24_MHD.png}\label{fig:convergenceDM}}
  \caption{Comparison of the effect of resolution on DM and MHD simualations. Here level4 curves (magenta) are compared to the mean and 2std (confirm) of the random-sampled curves from level3. For better comparison of the effect of resolution, the difference percent is plotted in green.  \textbf{bigger axes font}}
\end{figure}

Halo shape is heavily determined by accretion with environment \cite{shape relation with environment}. As environment may also be affected by resolution, this may have an effect on the halo shape. Also, resolution discrepancies may snowball through history.

\section{The shape's radial dependence}
One of the first results we obtained in this work is related to the evolution of the DM halo shape in terms of the radius at which it is sampled. We already expect from previous work that the shape does not remain constant along the radius \cite{Vera-Ciro et al 2011}. Specifically, we know that halos are gradually constructed from inner shells to outer shells through the accretion of matter from cosmic structures \cite{}. Inner shells are isolated from outter regions as a consequence of the Gauss law. Therefore, inner shells tend to conserve their shape. Outter shells, on the other side, are affected by the increasing gravitational potential form the inside of the halo, which makes them prone to scattering effects. This scattering of particles has a "rounding" effect on the outerskirts shape. For this reason, we expect on both simulations (MHD and DM) that halos are more triaxial on inner regions and more spherical at bigger radii. This effect has been corroborated on multiple cosmological simulations (Citas)\cite{} \\


\begin{figure}[!ht]
  \centering
  \subfloat[halo 27 DM shape at small radius]{\includegraphics[width=0.5\columnwidth]{./pics/MHD_Vs_DM/level4_DM_halo_27_inner.png}\label{fig:innerDM}}
  \hfill
  \subfloat[halo 27 DM shape at big radius]{\includegraphics[width=0.5\columnwidth]{./pics/MHD_Vs_DM/level4_DM_halo_27_outter.png}\label{fig:outterDM}}
  \hfill
  \subfloat[halo 27 MHD shape at small radius]{\includegraphics[width=0.5\columnwidth]{./pics/MHD_Vs_DM/level4_MHD_halo_27_inner.png}\label{fig:innerMHD}}
  \hfill
  \subfloat[halo 27 MHD shape at big radius]{\includegraphics[width=0.5\columnwidth]{./pics/MHD_Vs_DM/level4_MHD_halo_27_outter.png}\label{fig:outterMHD}}
  \caption{Example of the dependence of shape in terms of the radius. All graphics have matching orientation (which may not be the same) with their respective principal axes at the shown radii. The horizontal and vertical axes are aligned to the major and medium semi-axes respectively. }
\end{figure}

In figures \ref{fig:innerDM,fig:outterDM,fig:innerMHD,fig:outterMHD}, we present the MHD and DM-only version of a halo in which the rounding effect is specially evident for both degrees of realism. However, to eliminate any possible cualitative bias, we present a more detailed version of this effect in terms of the radius on figure \ref{fig:outterMHD}. There, we include all axial ratios, which clearly become closer to 1 (more spherical) for bigger radii. Besides the axial ratios, we included a measurement of triaxiality, namely $T=\frac{1-b/a}{1-c/a}$. This measurement $T$ tends towards unity when the medium-to-major axis ratio becomes equal to the minor-to-major ratio, i.e. when the halo becomes prolate. In the case where the medium axis is very close to the major axis, having a different minor axis, $T$ tends to a null value, i.e. when the halo tends to a oblate shape. In these terms, the halos are expected to be more prolate on the inside and more oblate on the outside. Even though the perfect spherical shape has a divergent/undefined $T$ value, prolate shapes are associated with triaxial characterizations and oblate shapes are identified as approximately spherical shapes. This, however, can be confirmed on the triaxiality plane where we can also demonstrate that this is in fact a global tendency on all halos.\\


\begin{figure}
\centering
{\includegraphics[width=1\columnwidth]{./pics/MHD_Vs_DM/level4_halo_27_DM_Vs_MHD.png}\label{fig:outterMHD}}
\caption{Semi-axial ratios and triaxiality $\frac{1-b/a}{1-c/a}$ as function of radius for semi-axes $a\geq b\geq c$. The MHD simulation (blue dotted line) shows ratios closer to $1$ than those from the DM-only (green solid line) simulation. The rounding effect with radius for each simulation separatedly is also well-appreciable in this graphic. The radial-rounding, as well as the gas-presence amplification can be evidenced on the triaxiality function. \textbf{show two examples or reorganize subplots to get a square figure, bigger label and axes font}}
\end{figure} 


For this reason, we present, on figure \ref{fig:}, the axial ratios on the plane $c/a Vs b/a$. In this case, each dot represents a specific halo from level4 simulations at a specific radius. Oblate halos are represented by the vertical line $x = 1$, prolate halos are identified on the identity line and spheres are exactly the point $(1,1)$. This gives us a broader idea of the evolution of the shape than a single number $T$ as well as the advantage of perfectly represent each halo as a dot to compare. Here, the tendency is clear for DM and MHD halos to get rounder with increasing radius. In fact, the difference in shape is so evident that it is possible to identify groups in case the radius label is lost.\\

\begin{figure}[!ht]
  \centering
  \subfloat[halo 24 MHD]{\includegraphics[width=0.5\columnwidth]{./pics/Convergence/rand_conv_halo24_DM.png}\label{fig:convergenceMHD}}
  \hfill
  \subfloat[halo 24 DM]{\includegraphics[width=0.5\columnwidth]{./pics/Convergence/rand_conv_halo24_MHD.png}\label{fig:convergenceDM}}
  \caption{Comparison of the effect of resolution on DM and MHD simualations. Here level4 curves (magenta) are compared to the mean and 2std (confirm) of the random-sampled curves from level3. For better comparison of the effect of resolution, the difference percent is plotted in green.  \textbf{bigger axes font}}
\end{figure}

So far, our result are in accordance with previous work on different kinds of simulations. Nonetheless, on the specific case of MW-like galaxy simulations, we have confirmed the expected tendence in an unprecedented statistical sample of 30 galaxies form Auriga compared to the 4 galaxies from the previous state-of-the art DM-only Aquarius simulations. Moreover, we confirmed that these results are sustained for the specific case of novel MHD MW-like galaxy simulations.\\

\subsection{The effect of gas on the halo shape}
The rounding effect with radius is specially evident in this example, but it is a general tendency over all. From this comparison we can notice that there is also a relation of this rounding effect with the presence of matter, which is to be expected. Unlike DM, gas collapses and generate disks which are much denser than DM at the scales where it forms. This amplifies scattering events and, if we apply the same logic, we would expect that the inner regions of the halo are more spherical when there is presence of DM. We expect the same for outter regions but this effect is expected to be more significant due to the stronger effect of the gravitational potential on the outter shells.\\



The general tendency can be better visualized on the triaxial plane (explain what is that).

\begin{figure}[!ht]
  \centering
  \subfloat[Level3 MHD Vs DM at inner regions]{\includegraphics[width=0.5\columnwidth]{./pics/Triaxial_Plane/Triaxiality_Inner_lvl3.png}\label{fig:innerTriaxial3}}
  \hfill
  \subfloat[Level3 MHD Vs DM at outter regions]{\includegraphics[width=0.5\columnwidth]{./pics/Triaxial_Plane/Triaxiality_Outer_lvl3.png}\label{fig:outterTriaxial3}}
  \hfill
  \centering
  \subfloat[Level4 MHD Vs DM at inner regions]{\includegraphics[width=0.5\columnwidth]{./pics/Triaxial_Plane/Triaxiality_Inner_lvl4.png}\label{fig:innerTriaxial4}}
  \hfill
  \subfloat[Level4 MHD Vs DM at outter regions]{\includegraphics[width=0.5\columnwidth]{./pics/Triaxial_Plane/Triaxiality_Outer_lvl4.png}\label{fig:outterTriaxial4}}
  \hfill
  \caption{General tendence on the triaxial plane $c/a Vs b/a$. Some observational constraints (stars and error-bar point) are plotted alongside our results}
\end{figure}

\section{Historical shape}
Explain why it is expected for the shape to remain more or less constant with time: non-collisional DM, gauss law effect.


\begin{figure}[!ht]
  \centering
  \subfloat[halo 16 DM]{\includegraphics[width=0.5\columnwidth]{./pics/Redshift/halo_16_level3_DM_Z.png}\label{fig:RedshiftDMgood}}
  \hfill
  \subfloat[halo 16 MHD]{\includegraphics[width=0.5\columnwidth]{./pics/Redshift/halo_16_level3_MHD_Z.png}\label{fig:RedshiftMHDgood}}
  \caption{Example of historic shape conservation in comoving coordinates.\textbf{improve colorbar}}
\end{figure}


\begin{figure}[!ht]
  \centering
  \subfloat[halo 21 DM]{\includegraphics[width=0.5\columnwidth]{./pics/Redshift/halo_21_level3_DM_Z.png}\label{fig:RedshiftDMbad}}
  \hfill
  \subfloat[halo 21 MHD]{\includegraphics[width=0.5\columnwidth]{./pics/Redshift/halo_21_level3_MHD_Z.png}\label{fig:RedshiftMHDbad}}
  \caption{Example of historic shape disruption in comoving coordinates. The consistency between MHD and DM implies some major-event like a close merger or a collision. The non-continuous red line corresponds to a very close moment of this merging event, which is amplified in MHD.  \textbf{improve colorbar}}
\end{figure}

\begin{figure}[!ht]
  \centering
  \subfloat[halo 16 MHD]{\includegraphics[width=0.5\columnwidth]{./pics/Redshift/halo_16_level3_MHD_Z_Triax.png}\label{fig:RedshiftDMTriax16}}
  \hfill
  \subfloat[halo 21 MHD]{\includegraphics[width=0.5\columnwidth]{./pics/Redshift/halo_21_level3_MHD_Z_Triax.png}\label{fig:RedshiftDMTriax21}}
  \caption{Historic shape Vs radial shape on the Triaxiality plane. The black line represents the radial profile at redshift 0. The colored connected dots represent the shape measured at the virial radius (physical coordinates) at a certain redshift (color)}
\end{figure}




\end{figure}


 
