\chapter{Conclusions}

In this work we use Allgood's method for shape calculation on the 30-sized set of DM-only and MHD MW-like simulations from the Auriga project to verify the results obtained by Vera-Ciro et al. 2011 about the shape of DM halos from Aquarius simulations and obtain more insight about how DM halos look on MW-like simulations. Auriga includes consistent models for energetic and accretion feedback from stars and Black Holes and the unique inclusion of magnetic fields. These models worked over a significant set of 30 galaxies evolved with the Arepo code, which solves the principal problems of the previous Computational Fluid Dynamys paradigms. All these properties make of Auriga one of the most advanced and precise simulations in actuality.\\ 

Our work is motivated by various factors. On one hand, obtaining insight on the halo shape of full-physics MW-like simulations may be applied to the improvement of constraints on the DM density field of our MW. This would result in a better comprehension of the behaviour and nature of DM itself. On the other hand, the novelty of gas models in these simulations makes them unique for the understanding on how the presence of visible matter affects the axial ratios of DM halos. Given the vast work on DM-only simulations, our results could be applied to make more realistic conclusions on these previous studies. Finally, the unique significant sample of 30 high-resolution galaxies from Auriga simulations makes our results statistically-supported.\\  

Taking this into account, we verify that, at $z=0$, DM halos from DM-only and MHD simulations are more oblate/spherical on outer regions and more prolate/triaxial on inner parts. We corroborate this effect in different manners, by obtaining the radial profile of the axial ratios, calculating a triaxiality indicator $T$ and presenting our results on the triaxial plane. Although our results were expected from the work on various cosmological simulations \cite{various}, our study is supported with an unprecedented sample of 30 level 4 resolution MW-like simulated galaxies.\\

Taking advantage of the parallel outcome of DM-only and MHD versions of the same galaxies on Auriga, we compare both versions to analyze the effect of the presence of gas on the shape of the DM halos. We find that gas affects the shape at all radii by makin the halo more spherical. Furthermore, we demonstrate that this rounding effect is more prominent on the outer regions of the halo. Although the general rounding effect due to the presence of matter is expected by previous studies on cosmological simulations \cite{}, our results are in conflict with previous work on the strength of this effect in terms of radius \cite{}. Usually, it is found that the rounding of halos is more evident on inner parts than on the outerskirts.\\

Vera-Ciro et al. deduced by inspection and showed that there is a correlation between the radial profile of the halo's axial ratios and the historical evolution at a determined radius. We corroborate this fact for DM-only simulations and show the reason for this correlation by calculating the radial shapes at comoving coordinates. We discover that the shape remains more or less unchanged in time once we account for the continuous rounding effect which is shown to be more prominent on the outerskirts of the halo, due to the continuous exposure to the inner gravitational potential. This is consistent with our results on the strength of the rounding effect of gas, which is also more intense for bigger radii.\\

We conclude our study by stating some interesting questions and proposing further studies on this matter.\\

First, our results are in general supported by previous works on cosmological simulations, with the exemption of the strength of the rounding effect of gas. In this sense it is of special interest to identify the causes of these discrepancies. We suspect that the principal source of these discrepancies may lie on the differing galaxy-formation models, the performance of the studies on different kinds of simulations and numerical effects of scattering. Previous work showed that the feedback efficiency diminishes the rouding effect by preventing matter to collapse at the center of the galaxy and produce stronger scattering \cite{}. In the case of Auriga simulations, to produce realistic MW-like disks, AGN feedback plays an important role in limiting the formation of stronge bulges \cite{Auriga}, which diminishes the strength of the effect based on conclusions from previous work. However, previous works are not limited to the study of MW-like galaxies but study galaxies in the general cosmological context \cite{studies} and inconsistencies may be due to specific effects of MW-like galaxies. Additionally, the rounding effect may be affected by resolution of previous work simulations. Here, we find that there is indeed a bigger resolution effect on MHD simulations due to the presence of gas, which may also contribute to these discrepancies. Nontheless, to confirm the causes of this conflic, further work must be performed.\\

Secondly, this work may be extended to the analysis of the impact of environmental structures on the shape of the DM halo. Here, following the work of Vera-Ciro et al. 2011, we must also analyze the relation of shapes with angular momentum and the specific orientation of the principal axes of the halo with respect to those determined by cosmic structures like fillaments. This could shed light on the effect of external structures on the DM halo shape, giving a more complete picture of how the the DM halos are shaped through history.\\


Finally, we could make use of the exeptional number of simulated galaxies to address statistical problems. For example, we could corroborate and improve theoretical models that predict the response of DM halos to the presence of matter, such as adiabatic contractions \cite{}. Adiabatic contractions arise from the assumption that the mass of gass increases at the center of the halo so slowly that we can consider that at any time, DM particles reach a stable orbit. Making use of adiabatic invariants, one could obtain a relation between the DM halo density and the gas density. This would support the advanced work on DM-only simulations by making them more observationally comparable by including the effects of gas, as well as making predictions of DM halos from the observed content of gas \cite{}.\\




 
 
