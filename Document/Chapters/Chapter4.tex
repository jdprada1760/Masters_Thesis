\chapter{Conclusions}


In this monograph we briefly reviewed the work of Vera-Ciro et al. (2011) \cite{Vera Ciro 2011} about the shape of DM halos on the MW-like DM-only simulations from Aquarius. We redproduced some of its results on the novel Auriga and verified that the general tendence from the outcome of our study on DM-only simulations still holds for MHD simulations. Furthermore, we compared the DM and MHD versions of each galaxy from Auriga simulations to analyze the effect of gas on the shape of each DM halo.\\

Specifically, we used Allgood's method for shape calculation to verify that, at $z=0$, DM halos from DM-only and MHD simulations are more oblate/spherical on outer regions and they more prolate/triaxial on inner parts. We verified this effect in different manners, by obtaining the radial profile of the axial ratios, calculating a triaxiality indicator $T$ and presenting our results on the triaxial plane. Although these results were expected from the work on various cosmological simulations \cite{various}, our study is supported with an unprecedented and significant sample of 30 level 4 resolution MW-like simulated galaxies.\\

Taking advantage of the parallel outcome of DM-only and MHD versions of the same galaxies, we could compare both versions to analyze the effect of the presence of gas on the shape of the DM halos. We found that gas in general affects the shape at all radii by rounding it. Furthermore, we verified that this rounding effect is more prominent on the outer regions. Although the general rounding effect due to the presence of matter is expected by previous studies on cosmological simulations \cite{}, our results are in conflict with previous work on the strength of this effect in terms of radius \cite{}, where it is calculated that the rounding of the halo is more evident on inner parts than on the outerskirts.\\

Vera-Ciro et al. deduced by inspection and showed that there is a correlation between the radial profile of the halo's axial ratios and their historical evolution. We verified this fact for DM-only simulations and showed the reason for this correlation by calculating the radial shapes at comoving coordinates. We discovered that the shape remains more or less unchanged once we account for the continuous rounding effect which is shown to be more prominent on the outerskirts of the halo, due to the continuos exposure to exposure to scattering effects. This also justifies our result that the rounding effect of gas is stronger for bigger radii, however, further work is needed to clarify the cause of discrepancies with previous studies.\\

This study could be extended to the analysis of the relation of environmental cosmological structures in the determination of the halo shapes and their angular momentum. Furthermore, we can take advantage of the significant size of the sample of galaxies from Auriga to obtain some statistics of our calculations and analyze to what extent our results are affected by random dispersion. We can also use this sample to constrain free parametres from theroretical models like adiabatic contraction \cite{}, which would be very important to predict the effect of gas on the density field of DM only halos.\\

 
 
