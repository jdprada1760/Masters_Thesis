\chapter{Conclusions}

Our work on the expected shape of DM halos is motivated by various factors. On one hand, we want to obtain some insight on the halo shape of full-physics MW-like simulations like Auriga \cite{auriga} that may be applied to the improvement of constraints on the DM density field of our MW. This would result in a better comprehension of the behaviour and nature of DM itself. On the other hand, the novelty of baryonic models in Auriga simulations, which we exploit, makes them unique for the understanding of the effect of visible matter on the axial ratios of DM halos. Given the vast work on DM-only simulations, our results could be applied to make more realistic conclusions on these previous studies by predicting the effect of non-simulated baryons. Finally, the unique significant sample of 30 high-resolution galaxies from the Auriga project makes our results statistically-supported and may be used to obtain better understanding of random biases that may affect works of this kind.\\  


In this work we use Allgood's method for shape calculation \cite{Allgood_et_al._2006} on the 30-sized set of DM-only and MHD MW-like simulations from the Auriga \cite{auriga} project to verify the results obtained by \cite[Vera-Ciro et al. 2011]{Vera-Ciro_et_al._2011} about the shape of DM halos from Aquarius simulations and obtain more insight about how DM halos behave on MW-like galaxy simulations. Auriga includes consistent models for energetic and accretion feedback from stars and Black Holes and the unique inclusion of magnetic fields. These models worked over a significant set of 30 galaxies evolved with the Arepo code \cite{arepo}, which solves the principal problems of the previous Computational Fluid Dynamics paradigms. All these properties make of Auriga one of the most advanced and precise set of simulations in actuality.\\ 


On this set of novel simulations we verify that, at $z=0$, DM halos from DM-only and MHD simulations are more spherical on outer regions ($\approx$100Kpc) and more triaxial on inner parts ($\approx$1Kpc). We corroborate this effect in different manners: by obtaining the radial profile of the axial ratios, calculating a triaxiality indicator $T$ and presenting our results on the triaxial plane $c/a$  Vs $b/a$. Although our results were expected from the work on various cosmological simulations \cite{Frenk_et_al._1988,Dubinski_and_Carlberg_1991,Warren_et_al._1992,Cole_and_Lacey_1996,Hayashi_et_al._2007,Bett_et_al._2007,Vera-Ciro_et_al._2011}, our study is supported by an unprecedented sample of 30 level 4 resolution MW-like simulated galaxies.\\

Taking advantage of the parallel outcome of DM-only and MHD versions of the same galaxies on Auriga, we compare both versions to analyze the effect of the presence of visible matter on the shape of the DM halos. We find that gas affects the shape at all radii by making the halo more spherical. Furthermore, we demonstrate that this rounding effect is more prominent on the inner regions of the halo where the presence of the axisymmetric galactic disk potential plays an important role. These results are also in agreement with previous studies on this subject \cite{Barnes_and_Hernquist_1996,Springel_et_al._2004,Bryan_et_al._2013}.\\

Vera-Ciro et al. deduced by inspection and showed that there is a correlation between the radial profile of the halo's axial ratios and their historical evolution at a determined radius. We corroborate this fact for DM-only simulations and show the reason for this correlation by calculating the radial shapes at comoving coordinates. We discover that the shape remains more or less unchanged in time once we account for the continuous rounding effect which is shown to be more prominent on the outskirts of the halo, due to the continuous exposure to the inner gravitational potential. This systematic tendency towards sphericity is, together with the pronounced rounding effect at the outskirts of the halo is traduced in a correlation of the historical shape at the virial radius with the radial profile at $z=0$.\\

We conclude our study by stating some interesting questions and proposing further studies on this matter. First, this work may be extended to the analysis of the impact of environmental structures on the shape of the DM halo where it is of special interest the study of the relation of shapes with angular momentum and the orientation of the principal axes of the halo with respect to those determined by cosmic structures and the galactic disk. Furthermore, we could make use of the exceptional number of simulated galaxies to address statistical problems such as the verification and improvement of theoretical models that predict the response of DM halos to the presence of matter, like adiabatic contractions \cite{Gnedin_et_al._2004}. \\




 
 
