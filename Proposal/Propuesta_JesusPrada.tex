\documentclass[12pt]{article}

\usepackage{graphicx}
\usepackage{epstopdf}
%\usepackage[english]{babel}
\usepackage[latin5]{inputenc}
\usepackage{hyperref}
\usepackage[left=3cm,top=3cm,right=3cm,nohead,nofoot]{geometry}
\usepackage{braket}
\usepackage{datenumber}
%\newdate{date}{10}{05}{2013}
%\date{\displaydate{date}}

\begin{document}

\begin{center}
\Huge
Dark matter halo shapes in the AURIGA Milkiway-like galaxy simulations

\vspace{3mm}
\Large Jesus David Prada Gonzalez

\large
201214619


\vspace{2mm}
\Large
Advisor: Jaime E. Forero-Romero\\
Coadvisor: Volker Springel
\normalsize
\vspace{2mm}

\today
\end{center}


\normalsize
\section{Introduction}

%Introducci�n a la propuesta de Monograf�a. Debe incluir un breve resumen del estado del arte del problema a tratar. Tambi�n deben aparecer citadas todas las referencias de la bibliograf�a (a menos de que se citen m�s adelante, en los objetivos o metodolog�a, por ejemplo)

Nowadays, the esence of Dark Matter (DM) remains unknown and is one of the biggest puzzles needed to understand the fundamentals that constitute our universe. As it is assumed by the nature of its discovery that it does not possess electromagnetic interactions, observing DM directly is practically impossible. However, due to its strong gravitational effect on the surrounding vissible matter (acceleration curves, weak lensing), its presence can be sensed and furthermore, its density field can be constrained.\\

In this context, it is of much interest to probe the density field of the DM that clusters (DM halo) around the Milky way as it can shed light in this fundamental enigma and it may have implications in many areas of physics. Specifically, the complete density field of the DM halo of our host galaxy is strong evidence to deduce many features of its formation history and evolution.\\

Many methods have been developed to constrain the shape of the Milky way's DM halo, ranging from the use of Jean's equations \cite{Loebman et al. 2012} to the satellite systems such as the Sagittarius stream and the Large Magellanic Cloud \cite{Vera-Ciro et al 2013, Deg & Widrow 2012, Law & Majewski 2010}. However, due to the difficulty in the observation of some specific sensitive details of our galaxy and its surrounding systems, many assumptions have to be done over these models, which produce considerably different results between each study. Then, it is safe to say that the constraints on the density field of the DM halo of the Milky way is still an open research topic in present-day astronomy. \\

Given these difficulties in observational astronomy and the lack of control over the state in some specific measurements of the system that is being studied, computational astrophysics comes in very handy as a method to support, confirm or even propose observations. Nevertheless, the solution of the measurement disadvantages in observational astronomy by computational astrophysics comes at a cost: the rise of numerical biases associated to the physics models and the resolution impediments by current-day computational power. In this sense, the study of simulations of astronomic or cosmological systems, as well as the research for reducing the aforementioned biases of computation, compose a very important field of study in modern astrophysics that serves as a compliment for observational measurements.\\

Recently, with the growth of computational power and the improvement of numerical models, the performance and further study of realistic simulations which trace the non-linear interactions of DM and baryonic components has been possible. These simulations can reproduce important features of our observable universe in a wide range of scales, from the cosmic star formation rate density and galaxy luminosity function in cosmological simulations to more specific features of Milky way type galaxies such as their stellar masses, rotation curves, star formation rates and metallicities. In these observations, the freedom in measurements and the control over the state of the systems and its observables, constitute the biggest advantages over observational astronomy.\\

In this context, the analysis of realistic simulations of Milky way-like galaxies, which has only been possible until very recently \cite{aquarius}, is of great importance to complement and perhaps give clues about details to have into account when probing the DM density field in observations regarding our galaxy. However, realistically reproducing the features of our Miklyway galaxy is not an easy task. It requires producing the correct initial conditions and not only having a sophisticated full-physics model to reproduce observables and a powerful computer, but to very carefully tune the free parameters of these models such as the ones associated to the many dissipation and feedback processes of baryonic matter to produce observables which are comparable to the observational data. This is why, before the arrival of realistic Milky way-like simulations such as Aquarius \cite{aquarius}, there was a generation of DM-only simulations which used the final state of the evolution of DM to reproduce, via semi-analytical models, the statistical features of the observable universe. These type of simulations have substantial information to work with, but may be biased in aspects regarding the historical relation between DM and baryonic matter, which is related to the question of our galaxy's DM density field \cite{relation DM baryons is important}. The task of incorporating baryonic matter in these simmulations is in fact so difficult that, even with the most correct prescriptions of that date, Aquarius is a recent set of just six Milky way-like galaxies, which can make any study performed on it of low statistical significance.\\

More recently, with the deveolpment of the latest and most accurate hydrodynamical code AREPO \cite{arepo} and the improvements of the physical numerial models regarding baryons, it has been possible the simulation of thirty Milky way-like galaxies in the project AURIGA \cite{auriga}. This code AREPO conciles the advantages and solves the flaws of the two paradigms of cosmology-oriented numerical hydrodynacs models namely Smoothed Particle Hydrodynamics (SPH) and Eulerian hydrodynamics with Adaptative Mesh Refinement (AMR). Furthermore, it can simulate magnetic fields, which is a novel feature in this type of simulations. Therefore, it becomes clear that the study of the DM density field on these simulations may produce a more complete insight when performed on these simulations due to the statistical significance of the sample which is also strong evidence of the big improvements of the baryonic state-of-the-art physics models.\\

In this order of ideas, the principal objective of this monography is to use the results of the pioneering AURIGA project \cite{AURIGA} to study the halo density field of these thirty galaxies obtaining statistically significant results which can then be compared to the state-of-the-art observations. Specifically, we will focus on the shape of the DM halo in function of its radius and its history, according to the guidelines stablished in a previous and similar study over the Aquarius simulations by Vera-Ciro et al. \cite{Vera-Ciro et al 2011}. The performance of this research on the AURIGA simulations is highly motivated by the fact that studies of the DM density field, like this have not been carried out on this nouvelle project. Hopefully, with this study we can elucidate some aspects about the history and relation of baryonic matter and DM in Milky way-like galaxies, which can then be extrapolated to the observational field of study, contributing in this way with our grain of sand in the trascendental enigma that represents DM to modern physics.\\


\section{General objectives}

%Objetivo general del trabajo. Empieza con un verbo en infinitivo.

Study the distribution of DM in Milky way-like galaxies.

\section{Specific objectives}

%Objetivos espec�ficos del trabajo. Empiezan con un verbo en infinitivo.

\begin{itemize}
	\item Perform the study of the distribution of DM in full-physics state-of-the art simulations.

	\item Study how the presence of gas affects the distribution of DM in full-physics simulations comparing results with DM-only simulations.
\end{itemize}

\section{Methodology}

%Exponer DETALLADAMENTE la metodolog�a que se usar� en la Monograf�a. 

%Monograf�a te�rica o computacional: �C�mo se har�n los c�lculos te�ricos? �C�mo se har�n las simulaciones? �Qu� requerimientos computacionales se necesitan? �Qu� espacios f�sicos o virtuales se van a utilizar?

%Monograf�a experimental: Recordar que para ser aprobada, los aparatos e insumos experimentales que se usar�n en la Monograf�a deben estar previamente disponibles en la Universidad, o garantizar su disponibilidad para el tiempo en el que se realizar� la misma. �Qu� montajes experimentales se van a usar y que material se requiere? �En qu� espacio f�sico se llevar�n a cabo los experimentos? Si se usan aparatos externos, �qu� permisos se necesitan? Si hay que realizar pagos a terceros, �c�mo se financiar� esto?

The student will perform the mentioned research individually with the periodic (semanal) support of his advisors through in-person meetings in the Astrophysics group when possible or electronic message interchange otherwise. In these meetings the student will obtain feedback of his work development and it will be decided if more time is necessary to discuss the partial results. The student has visited Heidelberg's Institute of Theoretical studies where he was an invited scientist sponsored by the Latinamerican Chinese European Galaxy Formation Network (LACEGAL) for one month \cite{LACEGAL}. Specifically, the student worked with the Theoretical Astrophysics (TAP) group which was under the leadership of Volker Springel. It was there where the main ideas of this project were born. Some other visits to HITS are planned during the course of this monography with the objective of obtaining in-person feedback from the co-advisor. These visits would also be fully funded by the LACEGAL project.\\

The proposed methodology has a strong computational component related to the analysis of high resolution galaxy simulations. The student has access, from his previously mentioned visit, to the computational cluster at HITS where the information of these simulations reside. He will at most need access to the Uniandes computational cluster. Furthermore, the revision of specialized bibliography is indispensable.\\


\section{Cronogram}

\begin{table}[htb]
	\begin{tabular}{|c|cccccccccccccccc| }
	\hline
	Tareas $\backslash$ Semanas & 1 & 2 & 3 & 4 & 5 & 6 & 7 & 8 & 9 & 10 & 11 & 12 & 13 & 14 & 15 & 16  \\
	\hline
	1 & X & X & X & X & X & X &   &   &   &   &   &   &   &   &   &   \\
	2 & X & X & X & X & X & X & X & X & X &   &   &   &   &   &   &   \\
	3 &   &   &   &   &   &   &   & X & X & X & X & X & X & X & X & X \\
	4 &   &   &   &   &   &   &   &   &   &   &   & X & X & X & X & X \\
	5 &   &   &   &   &   &   &   &   &   & X & X & X & X & X & X & X \\
	9 &   &   &   &   & X & X & X & X & X &   &   &   & X & X & X & X \\

	
	Tareas $\backslash$ Semanas & 17 & 18 & 19 & 20 & 21 & 22 & 23 & 24 & 25 & 26 & 27 & 28 & 29 & 30 & 31 & 32  \\
	\hline
	4 & X & X & X & X & X & X &   &   &   &   &   &   &   &   &   &   \\
	6 &   &   & X & X & X & X & X & X &   &   &   &   &   &   &   &   \\
	7 &   &   &   &   &   & X & X & X & X & X & X &   &   &   &   &   \\
	8 &   &   &   &   &   &   &   &   & X & X & X & X & X &   &   &   \\
	9 & X & X & X & X & X & X & X & X & X & X & X & X & X & X & X & X \\


	\hline
	\end{tabular}
\end{table}
\vspace{1mm}

\begin{itemize}
	\item Tarea 1: Desarrollo de redes neuronales simples
	\item Tarea 2: Investigaci�n del dise�o y funcionamiento de redes neuronales
	\item Tarea 3: Investigaci�n de modelos te�ricos sobre redes neuronales 
	\item Tarea 4: Desarrollo de intuiciones o modelos verificables sobre el funcionamiento del aprendizaje de las redes
	\item Tarea 5: Preparaci�n de la presentaci�n de avance
	\item Tarea 6: Verificaci�n de las conclusiones de los modelos anteriores sobre las redes neuronales artificiales espec�ficas
	\item Tarea 7: Ampliaci�n a redes neuronales m�s complejas
	\item Tarea 8: Desarrollo de resultados y conclusiones del trabajo
	\item Tarea 9: Escritura del documento final
\end{itemize}

\section{People who know about this topic}

%Nombres de por lo menos 3 profesores que conozcan del tema. Uno de ellos debe ser profesor de planta de la Universidad de los Andes.

\begin{itemize}
	\item Alejandro Garcia (Universidad de los Andes)
	\item Benjamin Oostra (Universidad de los Andes)
	\item Beatriz Sabogal (Universidad de los Andes)
\end{itemize}


%\begin{thebibliography}{10}
%\end{thebibliography}
\bibliographystyle{unsrt}
\bibliography{Bibliography}



\section*{Firma del Director}
\vspace{1.5cm}

\section*{Firma del Estudiante}



\end{document} 